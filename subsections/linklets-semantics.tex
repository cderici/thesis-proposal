\subsubsection{Operational Semantics of Linklets}
\label{subsec:linklets-semantics}

In this section I present a formalism to briefly describe the
operational semantics of linklets, which is the basis of the
implementation on Pycket. The formalism is developed in the PLT Redex
language, therefore it is executable
\cite{felleisen2009semantics}. Randomized testing is used to verify
the model \cite{random-test:2009}, and a parallel test suite is used
to verify the Pycket implementation and the model with each other.

Only the parts of the formalism that are essential to the discussion
in this document are shown here, the presentation of the full
formalism and the semantics are deferred to the
dissertation.

\begin{wrapfigure}[14]{r}{0.45\textwidth}
  \vspace{-0.7cm}
  \footnotesize
%\begin{figure}[h!]
  \begin{mdframed}
    \begin{align*}
      L  &::= (\textbf{linklet}\:((imp\: \ldots)\: \ldots)\:(exp\: \ldots)\:l\dash top\:\ldots\:e)\\
      l\dash top &::= (\textbf{define\dash values}\:(x)\:e)\:|\:e \\
      imp &::= x\:|\:(x x) \tag*{[external internal]}\\
      exp &::= x\:|\:(x x) \tag*{[internal external]} \\
      e &::=\; x\; |\; v\; |\; (e\; e\; \ldots)\; |\; (\textbf{if}\; e\; e\; e)\; |\; (o\; e\; e)\; \\
      \; &|\; (\textbf{begin}\; e\; e\; \ldots)\; |\; (\textbf{lambda}\; (x_{\_!\_}\; \ldots)\; e)\\
      \; &|\; (\textbf{set!}\; x\; e)\; |\; (\textbf{raises}\; e)\\
      \; &|\; (\textbf{var\dash ref}\; x)\; |\; (\textbf{var\dash ref/no\dash check}\; x) \\
      \; &|\; (\textbf{var\dash set!}\; x\; e)  \\
      \; &|\; (\textbf{var\dash set/check\dash undef!}\; x\; e)
    \end{align*}
  \caption[hede]{Linklet Source Language \footnotemark}
  \label{fig:linklet-source}
  \end{mdframed}
%\end{figure}
\end{wrapfigure}

\footnotetext{\label{fnlabel}Some rules (e.g. evaluation contexts etc.) are omitted for space.}

Figure \ref{fig:linklet-source} shows the grammar of the linklet
language, which is a minimal subset of the Fully Expanded Racket
Programs extended with the \verb|linklet| form and toplevel
definitions. The sub-language that is used within the linklet body,
namely the Racket Core (RC) is basically a $\lambda$-calculus with a handful
of syntactic extensions. In addition to the basic forms such
as \racketcode{begin}, \racketcode{set!}, \racketcode{if} etc., RC has
some additional forms to manipulate linklet variables
(e.g. \emph{var-ref}), as shown in \verb|e| rule. % figure \ref{fig:racket-source}.

%% \begin{wrapfigure}[24]{r}{0.45\textwidth}
%%   \vspace{-0.8cm}
%%   \footnotesize
%% %\begin{figure}[h]
%%   \begin{mdframed}
%%     \begin{align*}
%%       e &::=\; x\; |\; v\; |\; (e\; e\; \ldots)\; |\; (\textbf{if}\; e\; e\; e)\; |\; (o\; e\; e)\; \\
%%       \; &|\; (\textbf{begin}\; e\; e\; \ldots)\; |\; (\textbf{lambda}\; (x_{\_!\_}\; \ldots)\; e)\\
%%       \; &|\; (\textbf{set!}\; x\; e)\; |\; (\textbf{raises}\; e)\\
%%       \; &|\; (\textbf{var\dash ref}\; x)\; |\; (\textbf{var\dash ref/no\dash check}\; x) \\
%%       \; &|\; (\textbf{var\dash set!}\; x\; e)  \\
%%       \; &|\; (\textbf{var\dash set/check\dash undef!}\; x\; e) \\
%%       v   &::=\; n\; |\; b\; |\; c\; |\; (void)\;|\; uninit \\ %;; values
%%       c   &::=\; (closure\; (x\; \ldots)\; e\; \rho) \\
%%       n   &::=\; number \\
%%       b   &::=\; true\; |\; false \\
%%       var &::=\; variable\dash not\dash otherwise\dash mentioned \\ %;; variables
%%       x,\; cell &::=\; variable\dash not\dash otherwise\dash mentioned \\ %;; variables
%%       o  &::=\; +\; |\; * |\; < \\
%%       \rho   &::=\; ((x\; var)\; \ldots) \tag*{[Environment]}\\
%%       \sigma   &::=\; ((var\; v)\; \ldots) \tag*{[Store]}\\
%%       E   &::=\; []\; |\; (v\; \ldots\; E\; e\; \ldots)\; |\; (o\; E\; e)\; |\; (o\; v\; E) \\
%%       \;  &|\; (\textbf{var\dash set!}\; x\; E)\; |\; (\textbf{set!}\; x\; E)\; \\
%%       \;  &|\; (\textbf{var\dash set/check\dash undef!}\; x\; E) \\
%%       \;  &|\; (\textbf{begin}\; v\; \ldots\; E\; e\; \ldots)\; |\; (\textbf{if}\; E\; e\; e)
%%     \end{align*}
%%     \caption{Racket Core Language (RC)}
%%     \label{fig:racket-source}
%%   \end{mdframed}
%% %\end{figure}
%% \end{wrapfigure}

Before a linklet is ready for instantiation, as part of the
\racketcode{compile-linklet} process, several passes over the linklet are
performed to ensure a correct information flow between the variables
in the run\dash time. For each import and export identifier (and
possibly their corresponding renamings), \emph{Import} and
\emph{Export} objects are created with the fresh references that
will be used to name the variables that are dynamically created during
the instantiation. Additionally, a pass on the linklet body collects
all the identifiers that are defined within the linklet body and all
the mutated identifiers.

Given a set of \emph{Import}s ($\mathit{C_I}$) and a set of
\emph{Export}s ($\mathit{C_E}$), a set of toplevel defined
identifiers ($\mathit{X_T}$) and a set of mutated identifiers
($\mathit{X_M}$), the compilation starts to process each expression in
the linklet body one by one, performing the following actions
depending on the type of expression:

\begin{wrapfigure}[14]{r}{0.45\textwidth}
  \vspace{-0.6cm}
  \footnotesize
  %\small
%\begin{figure}[h!]
  \begin{mdframed}
    \begin{align*}
      CL             &::=\; \bm{\Phi^C}(L)\\
      L\dash obj     &::= (\la\; c\dash imps \; c\dash exps \; l\dash top \ldots)\\
      \;           &\;|\; (\lb\; x\; l\dash top\; \ldots) \\
      LI             &::=\; (\textbf{linklet\dash instance}\; (x\; var)\; \ldots) \\
      c\dash imps    &::=\; ((imp\dash obj\; \ldots)\; \ldots)\\
      c\dash exps    &::=\; (exp\dash obj\; \ldots)\\
      imp\dash obj   &::=\; (\textbf{Import}\; x\; x\; x) \tag*{[id int ext]} \\
      exp\dash obj   &::=\; (\textbf{Export}\; x\; x\; x) \tag*{[id int ext]} \\
    \end{align*}
    \hfill \footnotesize $\Phi^C : \textbf{compile\dash linklet}$
    \caption[aaa]{Linklet Runtime Language \footref{fnlabel}}
    \label{fig:linklet-runtime}
  \end{mdframed}
%\end{figure}
\end{wrapfigure}

\begin{itemize}
  \small
\item For a `(\textbf{define-values} (x) e)` expression where $(\textbf{Export}\; x_{gen}\; x\; x_{ext}) \in \mathit{C_E}$ (i.e. x is exported), add an additional `(\textbf{var-set!} $x_{gen}$ x)` expression.
\item For a `(\textbf{set!} x e)` expression where $(\textbf{Export}\; x_{gen}\; x\; x_{ext}) \in \mathit{C_E}$ (i.e. x is exported), turn the expression into \\ `(\textbf{var-set/check-undef!} $x_{gen}$ e)`.
\item For any identifier x where $(\textbf{Import}\; x_{gen}\; x\; x_{ext}) \in \mathit{C_I}$  (i.e. x is imported), turn the identifier into \\ `(\textbf{var-ref/no-check} $x_{gen}$)`.
\item For any identifier x where $(\textbf{Export}\; x_{gen}\; x\; x_{ext}) \in \mathit{C_E}$ and either $x \in \mathit{X_M}$ or $x \notin \mathit{X_T}$, turn the identifier into \\ `(\textbf{var-ref} $x_{gen}$)`.
\item For any other expression e, recurse if e has subforms, return otherwise.
\end{itemize}

Figure \ref{fig:linklet-runtime} shows the grammar of the run\dash
time language for linklets, i.e. expressions that represent the
run\dash time objects used during an instantiation of linklets. Given a
linklet L, \emph{compile-linklet} produces a linklet object $\la$ that
is ready to be instantiated. $\la$ consists of \emph{Import} and
\emph{Export} objects for the identifiers denoting the imported and
exported variables, and the body expressions with potentially some
additional expressions for variable manipulations in the run\dash
time.

Each \emph{Import} and \emph{Export} object consists of two
identifiers that are internal and external with respect to the linklet
body (to accommodate renamings), and one fresh identifier for internal
use during the instantiation. The \emph{Import} and \emph{Export}
objects are processed as the first step of the instantiation. As we
detail below, references for the imported variables are collected from
the given import instances, and additional variables may be created
for exported variables depending on the target instance as we have
shown in the example in \secref{subsec:toplevel-example}. $\lb$
denotes a compiled linklet object after such a preparation step, with
which the actual instantition may begin. Note that $\lb$ contains only
a reference (to the target instance that is used during the
instantiation) and the body expressions, and doesn't contain the
import and export objects, as the instantiation uses the variables
that are already prepared for them. Finally
a \emph{linklet-instance} is a finite mapping from symbols to
variables.

Since the linklets are lambda\dash like binding forms and not capable
of starting a computation by themselves, we introduce a top\dash level
form, namely \racketcode{program}. A \racketcode{program} consists of
a set of given linklets to be loaded and a single top\dash level
expression to perform a computation. \figref{fig:linklet-program}
shows part of the grammar for \racketcode{program}s. We provided
the \racketcode{let-inst} form to name a linklet instance,
and \racketcode{seq} form to perform multiple computations in
sequence.

\begin{wrapfigure}[13]{r}{0.5\textwidth}
  \footnotesize
  \vspace{-0.5cm}
%\begin{figure}[h]
  \begin{mdframed}
    \begin{align*}
      p           &::=\; (\textbf{program}\; ((x\; L)\; \ldots)\; p\dash top) \\
      p\dash top  &::=\; v\; |\; (\textbf{let\dash inst}\; x\; p\dash top\;p\dash top) \\
      \;          &\;|\; (\textbf{seq}\; p\dash top\; \ldots) \\
      \;          &\;|\; \bm{\Phi^I}(l\dash ref\; x\; \ldots)\; |\; \bm{\Phi^I}(l\dash ref\; x\; \ldots\; \textbf{\#:t}\; x) \\
      \;          &\;|\; \bm{\Phi^V}(x\; x) \\
      l\dash ref &::=\; x\; |\; L\dash obj \\
      v           &::=\; ....\;|\;(v\;x) \\
    \end{align*}
    \hfill \footnotesize $\Phi^V : \textbf{instance\dash variable\dash value}\:,\: \Phi^I : \textbf{instantiate\dash linklet}$
    \caption[bbbb]{Linklet Program Source Language \footnotemark}
    \label{fig:linklet-program}
  \end{mdframed}
%\end{figure}
\end{wrapfigure}

\footnotetext{Some rules (e.g. evaluation contexts etc.) are omitted for space.}

As an example, we can represent the top-level repl example from the
previous section as a linklet \emph{program}, and simulate the
process. The (multi-step) reduction below demonstrate the compilation of each
linklet, representing the linklets produced by the expander at each
repl interaction. After the compilation, the linklets are substitiuted
in the body and the evaluation starts with reducing the
body. \figref{fig:reduction} shows the reduction relation, $\beta_p$, that
implements the evaluation, together with the reduction relation for
RC, $\beta_r$. For clarity, the reduction of the programs and the linklet
instantiation is defined together by the same relation ($\beta_p$) so that
the whole evaluation generating a single small-step sequence of
reductions from programs to values.

\begin{table}[h!]
  %\vspace{-0.3cm}
  \centering
  %\scriptsize
  \footnotesize
  \begin{tabular}{lc|c|c}
    &\textbf{program} & \textbf{$\rho$} & \textbf{$\sigma$} \\ \hline \hline
    &\begin{lstlisting}[mathescape]
(program ([l1 (linklet () (a k) (define-values (k) (lambda () a)) (void))]
           [l2 (linklet () (a) (define-values (a) 10) (void))]
           [l3 (linklet () (k) (k))])
  (let-inst t (make-instance)
     (seq ($\phi^I$ l1 #:t t) ($\phi^I$ l2 #:t t) ($\phi^I$ l3 #:t t))))
    \end{lstlisting} & [] & [] \\ \hline
    $\longrightarrow_{\beta p}*$&\begin{lstlisting}[mathescape]
(program ()
  (let-inst t (make-instance)
     (seq ($\phi^I$ (L$\alpha$ () ((Export a1 a a) (Export k1 k k))
                   (define-values (k) (lambda () (var-ref a1)))
                   (var-set! k1 k) (void)) #:t t)
           ($\phi^I$ (L$\alpha$ () ((Export a1 a a))
                   (define-values (a) 10) (var-set! a1 a) (void)) #:t t)
           ($\phi^I$ (L$\alpha$ () ((Export k1 k k)) ((var-ref k1))) #:t t))))
    \end{lstlisting} & [] & [] \\ \hline
  \end{tabular}
\end{table}

After the compilation and loading phase, all the linklet objects are
in $\la$ form and ready to be instantiated. Recall that a linklet
instantiation has two modes; \textbf{i)} regular instantiation, where
a new instance is being created and returned, and \textbf{ii)}
targeted instantiation, where the given target instance is used during
the instantiation. Since the only difference between the two modes is
the return value, for simplicity in the model we remove the
distinction by turning the regular instantiation into a targeted
instantiation with a fresh empty instance. The result of both the
instantiation is then a pair of the last expression's value and the
target instance being used.

At the start of instantiation, the imported variables are collected
from the given linklet instances by simply going through each set of
\emph{Import} objects in the $\la$ and the corresponding linklet
instance that provides the variables. The collected variable
references are then put into the environment with the names in the
corresponding \emph{Import} objects that were generated for the
variables during the compilation. Recall that a linklet may export a
variable despite having no corresponding definition for it. This is
achieved by for each \emph{Export} object in $\la$ grabbing the
variable from the target instance if it has a mapping for it and
making it available in the environment. Otherwise a fresh variable is
created and its mapping is placed into the environment and into the
target instance as well.

%% \begin{figure}[h!]
%%   \centering
%%   \footnotesize
%% %\begin{table}
%%   \begin{tabular}{lc|c|c}
%%     &\textbf{program} & \textbf{$\rho$} & \textbf{$\sigma$} \\ \hline \hline
%%     $\longrightarrow_{\beta p}$&\begin{lstlisting}[mathescape]
%% (program ()
%%   (seq ($\phi^I$ (L$\alpha$ () ((Export a1 a a) (Export k1 k k))
%%                 (define-values (k) (lambda () (var-ref a1)))
%%                 (var-set! k1 k) (void)) #:t t)
%%         ($\phi^I$ (L$\alpha$ () ((Export a1 a a))
%%                 (define-values (a) 10) (var-set! a1 a) (void)) #:t t)
%%         ($\phi^I$ (L$\alpha$ () ((Export k1 k k)) ((var-ref k1))) #:t t))))
%%     \end{lstlisting} & [] & \thead{[t $\rightarrow$ (linklet-instance), \\li $\rightarrow$ (linklet-instance)]} \\ \hline
%%     $\longrightarrow_{\beta p}$&\begin{lstlisting}[mathescape]
%% (program ()
%%   (seq ($\phi^I$ (L$\beta$ t (define-values (k) (lambda () (var-ref a1)))
%%                       (var-set! k1 k) (void)))
%%         ($\phi^I$ (L$\alpha$ () ((Export a1 a a))
%%                 (define-values (a) 10) (var-set! a1 a) (void)) #:t t)
%%         ($\phi^I$ (L$\alpha$ () ((Export k1 k k)) ((var-ref k1))) #:t t))))
%%     \end{lstlisting} & \thead{[k1 $\rightarrow$ $var_k$, \\a1 $\rightarrow$ $var_a$]} & \thead{[$var_a$ $\rightarrow$ uninit,\\ $var_k$ $\rightarrow$ uninit,\\ t $\rightarrow$ (linklet-instance \\ (a $var_a$) (k $var_k$)), \\li $\rightarrow$ (linklet-instance)]} \\ \hline
%%   \end{tabular}
%%   \caption{The first linklet's Import and Export objects are processed.}
%% %\end{table}
%% \end{figure}

After processing the import and export variables, the $\la$ object is
reduced into $\lb$, which triggers the evaluation of the expressions
in the linklet body. The linklet compilation makes sure that every
variable reference within the body uses the appropriate variable, and
at the beginning of the instantiation the preparation of the import
and export variables explained above provides the actual variables in
the environment to be modified and used. The reduction then proceeds
with the evaluation of the body expressions one-by-one. After all the
expressions are processed, a pair of the last expression's value and
the reference to the target instance is returned. The reduction of the
program continues by similarly instantiating the other two linklets
and finally returning the last value in the
\verb|seq| form.


\begin{figure}[h!]
  \footnotesize
  \begin{mdframed}
    \begin{align*}
      \rcinput{x} \rcrel \rcoutput{\sigma[\rho[x]]}{\rho}{\sigma} \\
      \rcinput{(\textbf{var\dash ref}\; x))} \rcrel \rcoutput{v}{\rho}{\sigma}\; \where x \in \rho, \rho[x] \in \sigma, v = \sigma[\rho[x]], v \neq uninit\\
      \rcinput{(\textbf{var\dash ref/no\dash check}\; x))} \rcrel \rcoutput{v}{\rho}{\sigma}\; \where v = \sigma[\rho[x]] \\
      \rcinput{(\textbf{var\dash set!}\; x\; v)} \rcrel \rcoutput{(void)}{\rho}{\sigma_1} \where \sigma_1 = \sigma[\rho[x]\rightarrow v]\\
      \rcinput{(\textbf{var\dash set/check\dash undef!}\; x\; v)} \rcrel \rcoutput{(void)}{\rho}{\sigma_1} \where \sigma_1 = \sigma[\rho[x]\rightarrow v], x \in \rho\\
      \rcinput{(\textbf{set!}\; x\; v)} \rcrel \rcoutput{(void)}{\rho}{\sigma_1} \where \sigma_1 = \sigma[\rho[x]\rightarrow v]\\
      \\
      \redinput{\bm{\Phi^V}(x_{li},\; x)} \rel \redoutput{v}{\rho}{\sigma} \where v\; =\; \sigma[(\sigma[x_{li}])[x]] \\
      %% \redinput{(\textbf{let\dash inst}\; x\; (v\;x_i)\; p\dash top)} \rel \redoutput{p\dash top}{\rho}{\sigma_1} \where \sigma_1 = \sigma[x\rightarrow \sigma[x_i]] \\
      %% \redinput{(\textbf{seq}\; v_{1}\; \ldots\; v_{n})} \rel \redoutput{v_{n}}{\rho}{\sigma} \\
      \redinput{(\bm{\Phi^I}\; (\lb\; x\; v\; \ldots\; v_{l}))} \rel \redoutput{(v_{l}\; x)}{\rho}{\sigma} \\
      EP\;\llbracket (\bm{\Phi^I}(\lb\; x_{t}\; v_p\; \ldots\; (\textbf{define\dash values}\; (x)\; v)\; l\dash top\; \ldots)) \rrbracket, \rel \redoutput{\bm{\Phi^I}(\lb\; x_{t}\; v_p\; \ldots\; l\dash top\; \ldots)}{\rho_1}{\sigma_1} \\
      \;          &\;\; \where \rho_1 = \rho[x\rightarrow cell],\; \sigma_1 = \sigma[cell\rightarrow v] \\
      \redinput{(\bm{\Phi^I}(\la\; c\dash imps \; c\dash exps \; l\dash top \ldots)\; LI\; \ldots)} \rel \redoutput{(\bm{\Phi^I}(\la\; c\dash imps \; c\dash exps \; l\dash top\; \ldots)\; LI\; \ldots\; \textbf{\#:t}\; x_{t})}{\rho}{\sigma_1} \\
      \;          &\;\; \where x_t \notin \mathit{dom}(\sigma), \sigma_1 = \sigma[x_t\rightarrow (linklet\dash instance)] \\
      EP\;\llbracket (\bm{\Phi^I}(\la\; c\dash imps \; c\dash exps \; l\dash top \ldots)\; LI\; \ldots\; \textbf{\#:t}\; x_{t}) \rrbracket, \rho, \sigma\; \rel \redoutput{(\bm{\Phi^I}(\lb\; x_{t}\; l\dash top\; \ldots))}{\rho_2}{\sigma_1} \\
      \; &\;\; \where \rho_1 = \mathit{V}^I(c\dash imps, (LI\; \ldots), \rho) \\
      \;          &\;\; \hspace{0.4cm} (\rho_2, \sigma_1) = \mathit{V}^E(c\dash exps, x_t, \rho_1, \sigma) \\
      (\textbf{program}\; ((x\;L),(x_1\;L_1)\;\ldots)\;p\dash top), \rho, \sigma &\longrightarrow_{\beta p}\; (\textbf{program}\; ((x_1\;L_1)\;\ldots)\;p\dash top[x := \bm{\Phi^C}(L)]), \rho, \sigma \\ \\
      eval^{\longrightarrow_{\beta p}}(p) &=\; v\; \textbf{if}\; p, (), () \twoheadrightarrow_{\beta p\cup\beta r} \; (\textbf{program}\; ()\;(v\;\_)) \\ \\
      %% \mathit{V}^I : c\dash imps \times (LI\;\ldots) \times \rho &\longrightarrow \rho \\
      %% \mathit{V}^I((\overline{(\overline{(\textbf{Import}\; x_{id}\;x_{int}\;x_{ext})})_n}), (\overline{LI_n}), \rho) &= \rho\overline{[x_{id}\rightarrow LI_n[x_{ext}]]}\\ \\
      %% \mathit{V}^E : c\dash exps \times x \times \rho \times \sigma &\longrightarrow \rho \times \sigma \\
      %% \mathit{V}^E((exp\dash obj\;\ldots), x_t, \rho, \sigma) &= \mathit{P}(exp\dash obj, x_t, \sigma[x_t], \rho, \sigma)\;\ldots\\
      %% \mathit{P}((\textbf{Export}\;x_{id}\;x_{int}\;x_{ext}), x_t, LI_t, \rho, \sigma) &= \begin{cases*}
      %%   \rho_1, \sigma & if $\;x_{ext} \in dom(LI_t)$, \where $\rho_1 = \rho[x_{id}\rightarrow LI_t[x_{ext}]]$ \\
      %%   \rho_1, \sigma_1 & if $\;x_{ext} \notin dom(LI_t)$, \where $\rho_1 = \rho[x_{id}\rightarrow var_{new}]$\\
      %%   \;        &\; \hspace{2.6cm} $var_{new} \notin \sigma$ \\
      %%   \;        &\; \hspace{2.6cm} $\sigma_1 = \sigma[var_{new}\rightarrow uninit,$ \\
      %%     \;        &\; \hspace{3.2cm} $x_t\rightarrow LI_t[x_{ext}\rightarrow var_{new}]]$
      %% \end{cases*} \\
    \end{align*}
    \centering
    \footnotesize $\Phi^V : \textbf{instance\dash variable\dash value}\:,\: \Phi^I : \textbf{instantiate\dash linklet},\:\Phi^C : \textbf{compile\dash linklet}$

    \footnotesize $\mathit{V}^I : \textbf{get  import variables},\: \mathit{V}^E : \textbf{create variables for exports}$
    \caption[ccc]{Reduction Relation \footnotemark}
    \label{fig:reduction}
  \end{mdframed}
\end{figure}

\begin{adjustbox}{center}
%\begin{table} %[h!]
  \footnotesize
%\begin{table}
  \begin{tabular}{lc|c|c}
    &\textbf{program} & \textbf{$\rho$} & \textbf{$\sigma$} \\ \hline \hline
    $\longrightarrow_{\beta p}$&\begin{lstlisting}[mathescape]
(program ()
  (seq ($\phi^I$ (L$\alpha$ () ((Export a1 a a) (Export k1 k k))
                (define-values (k) (lambda () (var-ref a1))) (var-set! k1 k) (void)) #:t t)
        ($\phi^I$ (L$\alpha$ () ((Export a1 a a))
                (define-values (a) 10) (var-set! a1 a) (void)) #:t t)
        ($\phi^I$ (L$\alpha$ () ((Export k1 k k)) ((var-ref k1))) #:t t))))
    \end{lstlisting} & [] & \thead{[t $\rightarrow$ (linklet-instance), \\li $\rightarrow$ (linklet-instance)]} \\ \hline
    $\longrightarrow_{\beta p}$&\begin{lstlisting}[mathescape]
(program ()
  (seq ($\phi^I$ (L$\beta$ t (define-values (k) (lambda () (var-ref a1))) (var-set! k1 k) (void)))
        ($\phi^I$ (L$\alpha$ () ((Export a1 a a))
                (define-values (a) 10) (var-set! a1 a) (void)) #:t t)
        ($\phi^I$ (L$\alpha$ () ((Export k1 k k)) ((var-ref k1))) #:t t))))
    \end{lstlisting} & \thead{[k1 $\rightarrow$ $var_k$, \\a1 $\rightarrow$ $var_a$]} & \thead{[$var_a,var_k$ $\rightarrow$ uninit,\\ t $\rightarrow$ (linklet-instance \\ (a $var_a$) (k $var_k$)), \\li $\rightarrow$ (linklet-instance)]} \\ \hline
    %&\textbf{program} & \textbf{$\rho$} & \textbf{$\sigma$} \\ \hline \hline
    $\longrightarrow_{\beta p}$&\begin{lstlisting}[mathescape]
(program ()
  (seq ($\phi^I$ (L$\beta$ t (var-set! k1 k) (void)))
        ($\phi^I$ (L$\alpha$ () ((Export a1 a a))
                (define-values (a) 10) (var-set! a1 a) (void)) #:t t)
        ($\phi^I$ (L$\alpha$ () ((Export k1 k k)) ((var-ref k1))) #:t t))))
    \end{lstlisting} & \thead{[k $\rightarrow$ $cell_1$,\\k1 $\rightarrow$ $var_k$,\\a1 $\rightarrow$ $var_a$]} & \thead{[$cell_1$ $\rightarrow$ closure,\\$var_a,var_k$ $\rightarrow$ uninit,\\t $\rightarrow$ (linklet-instance\\(a $var_a$) (k $var_k$)),\\li $\rightarrow$ (linklet-instance)]} \\ \hline
    $\longrightarrow_{\beta r}$&\begin{lstlisting}[mathescape]
(program ()
  (seq ($\phi^I$ (L$\beta$ t (var-set! k1 (closure () (var-ref a1) ((k1 $var_k$) (a1 $var_a$)))) (void)))
        ($\phi^I$ (L$\alpha$ () ((Export a1 a a))
                (define-values (a) 10) (var-set! a1 a) (void)) #:t t)
        ($\phi^I$ (L$\alpha$ () ((Export k1 k k)) ((var-ref k1))) #:t t))))
    \end{lstlisting} & \thead{[k $\rightarrow$ $cell_1$,\\k1 $\rightarrow$ $var_k$,\\a1 $\rightarrow$ $var_a$]} & \thead{[$cell_1$ $\rightarrow$ closure,\\$var_a,var_k$ $\rightarrow$ uninit,\\t $\rightarrow$ (linklet-instance\\(a $var_a$) (k $var_k$)),\\li $\rightarrow$ (linklet-instance)]} \\ \hline
    $\longrightarrow_{\beta r}$&\begin{lstlisting}[mathescape]
(program ()
  (seq ($\phi^I$ (L$\beta$ t (void)))
        ($\phi^I$ (L$\alpha$ () ((Export a1 a a))
                (define-values (a) 10) (var-set! a1 a) (void)) #:t t)
        ($\phi^I$ (L$\alpha$ () ((Export k1 k k)) ((var-ref k1))) #:t t))))
    \end{lstlisting} & \thead{[k $\rightarrow$ $cell_1$,\\k1 $\rightarrow$ $var_k$,\\a1 $\rightarrow$ $var_a$]} & \thead{[$cell_1$ $\rightarrow$ closure,\\$var_a$ $\rightarrow$ uninit,\\$var_k$ $\rightarrow$ closure,\\t $\rightarrow$ (linklet-instance\\(a $var_a$) (k $var_k$)),\\li $\rightarrow$ (linklet-instance)]} \\ \hline
    $\longrightarrow_{\beta p}$&\begin{lstlisting}[mathescape]
(program ()
  (seq ((void) t)
        ($\phi^I$ (L$\alpha$ () ((Export a1 a a))
                (define-values (a) 10) (var-set! a1 a) (void)) #:t t)
        ($\phi^I$ (L$\alpha$ () ((Export k1 k k)) ((var-ref k1))) #:t t))))
    \end{lstlisting} & \thead{[k $\rightarrow$ $cell_1$,\\k1 $\rightarrow$ $var_k$,\\a1 $\rightarrow$ $var_a$]} & \thead{[$cell_1$ $\rightarrow$ closure,\\$var_a$ $\rightarrow$ uninit,\\$var_k$ $\rightarrow$ closure,\\t $\rightarrow$ (linklet-instance\\(a $var_a$) (k $var_k$)),\\li $\rightarrow$ (linklet-instance)]} \\ \hline
  \end{tabular}
  %\caption{The first linklet is being instantiated. (Only showing the essential reduction steps)}
%\end{table}
%\end{figure}
\end{adjustbox}

%% \begin{figure}[h!]
%%   \footnotesize
%%   \begin{mdframed}
%%     \begin{align*}
%%       \rcinput{x} \rcrel \rcoutput{\sigma[\rho[x]]}{\rho}{\sigma} \\
%%       \rcinput{(\textbf{var\dash ref}\; x))} \rcrel \rcoutput{v}{\rho}{\sigma} \\
%%       \;    &\; \hspace{2cm} \where x \in \rho, \rho[x] \in \sigma, v = \sigma[\rho[x]]\; \&\; v \neq uninit\\
%%       \rcinput{(\textbf{var\dash ref/no\dash check}\; x))} \rcrel \rcoutput{v}{\rho}{\sigma}\; \where v = \sigma[\rho[x]] \\
%%       \rcinput{(\textbf{var\dash set!}\; x\; v)} \rcrel \rcoutput{(void)}{\rho}{\sigma_1} \where \sigma_1 = \sigma[\rho[x]\rightarrow v]\\
%%       \rcinput{(\textbf{var\dash set/check\dash undef!}\; x\; v)} \rcrel \rcoutput{(void)}{\rho}{\sigma_1} \where \sigma_1 = \sigma[\rho[x]\rightarrow v], x \in \rho\\
%%       \rcinput{(\textbf{set!}\; x\; v)} \rcrel \rcoutput{(void)}{\rho}{\sigma_1} \where \sigma_1 = \sigma[\rho[x]\rightarrow v]\\
%%       \rcinput{(\textbf{lambda}\; (x\; \ldots)\; e)} \rcrel \rcoutput{(closure\; (x\; \ldots)\; e\; \rho)}{\rho}{\sigma} \\
%%       \; &\ldots\; \\
%%       \rcinput{(\textbf{begin}\; v\; \ldots\; v_l)} \rcrel \rcoutput{v_l}{\rho}{\sigma} \\
%%       \rcinput{(\textbf{if}\; v\; e_1\; e_2)} \rcrel \rcoutput{e}{\rho}{\sigma} \where e = \begin{cases*}
%%         e_2 & if $v = false$ \\
%%         e_1 & otherwise
%%       \end{cases*}\\
%%       \rcinput{(o\; v_1\; v_2)} \rcrel \rcoutput{\delta(v_1,v_2)}{\rho}{\sigma} \\
%%       \rcinput{((closure\; (x\; \ldots)\; e\; \rho_c)\;v\;\ldots)} \rcrel \rcoutput{e}{\rho_1}{\sigma_1} \\
%%       \;     &\; \hspace{1cm} \where \rho_1 = \rho_c\overline{[x\rightarrow cell]}, \sigma_1 = \sigma\overline{[cell\rightarrow v]}, cell \notin \sigma
%%     \end{align*}
%%     \caption{Reduction Relation for Racket Core}
%%     \label{fig:rc-reduction}
%%   \end{mdframed}
%% \end{figure}

\footnotetext{Some reduction rules (e.g. closure application etc.) are omitted for space.}
