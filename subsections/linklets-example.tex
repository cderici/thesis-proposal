\subsubsection{Extended Example: Top-level REPL via Linklets}
\label{subsec:toplevel-example}

This section presents an extended example that demonstrates
\textbf{i)} how linklets are evaluated, and \textbf{ii)} how the
top-level interaction is implemented through the linklet
variables. First we will see a general overview of the mechanism, then
we will state the example in the formalism that's shown in
\secref{subsec:linklet-semantics} to see the essential details.

\begin{wrapfigure}[9]{r}{0.3\textwidth}
  %\vspace{-0.5cm}
  \begin{mdframed}
    \begin{verbatim}
> (define k (lambda () a))
> (define a 10)
> (k)
10
\end{verbatim}
    \caption{Top-level Example}
    \label{fig:toplevel-interaction}
  \end{mdframed}
\end{wrapfigure}

The top-level itself is quite complicated and beyond the scope of our
study. The example here aims to demonstrate the essential logic behind
the linklet variables that implement the functionalities like the
top-level repl\footnote{Read-Eval-Print Loop}.

The implementation of the top-level in the expander provides a
specially handled linklet instance, namely the \emph{top-level} that
contains a reference to the namespace and all the defined variables,
acting like a top-level environment. Each expression entered in the
repl is evaluated by inserting in a linklet that imports the necessary
bindings for expansion bookkeeping, compiling and instantiating with
the \emph{top-level} as the target instance. For clarity here we only
show the run-time linklets among the generated bundles and simplify
the internals to only deal with the essentials.

As our example, let's consider the interactions in
\figref{fig:toplevel-interaction}. The key point here is that the
variable "a" is defined \emph{after} it is used within the function
"k". Let's see what happens at each step:

\paragraph{$>$ (define k (lambda () a))}

As we briefly explained earlier, this expression is put into a linklet
body, then the linklet is instantiated over the \emph{top-level}
instance. Recall that a linklet instance is a finite mapping from
symbols to variables. Initially our target instance is empty.

\begin{figure}[h!]
  \small
  \begin{subfigure}[b]{0.25\textwidth}
    \begin{mdframed}
\begin{verbatim}
(linklet (...) (a k)
  (define-values (k)
    (lambda () a)))
\end{verbatim}
    \end{mdframed}
    \caption{Linklet \\ before instantiate}
    \label{fig:1}
  \end{subfigure}
  %
  \begin{subfigure}[b]{0.38\textwidth}
    \begin{mdframed}
\begin{verbatim}
(linklet (...) ((a0 a) (k0 k))
  (define-values (k)
    (lambda ()
      (variable-ref (LinkletVar a0))))
  (variable-set! (LinkletVar k0) k))
\end{verbatim}
    \end{mdframed}
    \caption{Compiled \\ Linklet}
    \label{fig:2}
  \end{subfigure} \hfill
    %
  \begin{subfigure}[b]{0.15\textwidth}
    \begin{mdframed}
      \begin{align*}
        a_0 &\rightarrow var_a \\
        k_0 &\rightarrow var_k \\
      \end{align*}
    \end{mdframed}
    \caption{Environment \\ after compile}
    \label{fig:2}
  \end{subfigure}
  %
  \begin{subfigure}[b]{0.15\textwidth}
    \begin{mdframed}
      \begin{align*}
        a &\rightarrow var_a \\
        k &\rightarrow var_k \\
      \end{align*}
    \end{mdframed}
    \caption{Target \\ after instantiate}
    \label{fig:2}
  \end{subfigure}
\end{figure}

The most essential point in this example is the handling of the
undefined identifier "a". When the expander generates a linklet
containing an unidentified identifier, it puts the identifier among
the exported variables of the linklet (despite having no corresponding
definition for it), allowing it to be handled via the linklet
variables. During the compilation of the linklet, a new linklet
variable is created for it with the special value
\emph{uninitialized}, and when the binding for the identifier is
needed, it will be dynamically provided through the target,
essentially forming a low-level dynamic scope. Note that we have the
variable mappings for both "a" and the "k" within the target instance
after the instantiation.

\paragraph{$>$ (define a 10)}

When this new expression is inserted in the repl, the expander again
generates a linklet, then compiles and instantiates it over the sam
\emph{top-level} target instance.

\vspace{-0.5cm}

\begin{figure}[h!]
  \small
  \begin{subfigure}[b]{0.26\textwidth}
    \begin{mdframed}
\begin{verbatim}
(linklet (...) (a)
  (define-values (a) 10))
\end{verbatim}
    \end{mdframed}
    \caption{Linklet \\ before instantiate}
    \label{fig:1}
  \end{subfigure}
  %
  \begin{subfigure}[b]{0.38\textwidth}
    \begin{mdframed}
\begin{verbatim}
(linklet (...) ((a1 a))
  (define-values (a) 10)
  (variable-set! (LinkletVar a1) a))
\end{verbatim}
    \end{mdframed}
    \caption{Compiled \\ Linklet}
    \label{fig:2}
  \end{subfigure} \hfill
  \begin{subfigure}[b]{0.15\textwidth}
    \begin{mdframed}
      \begin{align*}
        a_1 &\rightarrow var_a \\
      \end{align*}
    \end{mdframed}
    \caption{Environment \\ after compile}
    \label{fig:2}
  \end{subfigure}
  \begin{subfigure}[b]{0.15\textwidth}
    \begin{mdframed}
      \begin{align*}
        a &\rightarrow var_a \\
        k &\rightarrow var_k \\
      \end{align*}
    \end{mdframed}
    \caption{Target \\ after instantiate}
    \label{fig:2}
  \end{subfigure}
\end{figure}

Note that the compile-linklet inserted the \emph{variable-set!}
expression in the body to set the value of the variable bound by the
external name "a". This variable exists in two possible ways. When the
exported variables are processed at the start of the instantiation, if
the target instance doesn't have a variable mapping for "a", then a
new variable will be created (as in the previous step) with the
\emph{uninitialized} value, otherwise if the target does have a
mapping for "a" (which it does in this case), then target's variable
will be used during the instantiation, and will be set by
\emph{variable-set!}.

\paragraph{$>$ (k)}

At this step, we have everything we need to evaluate this expression
thanks to the \emph{top-level} target instance.

\begin{figure}[h!]
  \small
  \begin{subfigure}[b]{0.25\textwidth}
    \begin{mdframed}
\begin{verbatim}
(linklet (...) (k) (k))
\end{verbatim}
    \end{mdframed}
    \caption{Linklet \\ before instantiate}
    \label{fig:1}
  \end{subfigure}
  %
  \begin{subfigure}[b]{0.38\textwidth}
    \begin{mdframed}
\begin{verbatim}
(linklet (...) ((k1 k))
  ((variable-ref (LinkletVar k1))))
\end{verbatim}
    \end{mdframed}
    \caption{Compiled \\ Linklet}
    \label{fig:2}
  \end{subfigure} \hfill
    %
  \begin{subfigure}[b]{0.15\textwidth}
    \begin{mdframed}
      \begin{align*}
        k_1 &\rightarrow var_k \\
      \end{align*}
    \end{mdframed}
    \caption{Environment \\ after compile}
    \label{fig:2}
  \end{subfigure}
  %
  \begin{subfigure}[b]{0.15\textwidth}
    \begin{mdframed}
      \begin{align*}
        a &\rightarrow var_a \\
        k &\rightarrow var_k \\
      \end{align*}
    \end{mdframed}
    \caption{Target \\ after instantiate}
    \label{fig:2}
  \end{subfigure}
\end{figure}

Now let's take a look at the details behind the scenes. We first
represent the interactions as a linklet \emph{program} as introduced
in \secref{subsec:linklet-semantics}. We write each expression in the
interactions as they will be generated by the expander at each step,
load them in the program and then instantiate one by one over the same
target instance, which is initially an empty instance.

We start by compiling the linklets. For simplicity in the model, we
substitute the compiled linklets in the body of the program.

\begin{table}[h!]
  \centering
  \footnotesize
  \begin{tabular}{lc|c|c}
    &\textbf{program} & \textbf{$\rho$} & \textbf{$\sigma$} \\ \hline \hline
    &\begin{lstlisting}[mathescape]
(program ([l1 (linklet () (a k) (define-values (k) (lambda () a)) (void))]
           [l2 (linklet () (a) (define-values (a) 10) (void))]
           [l3 (linklet () (k) (k))])
  (let-inst t (make-instance)
     (seq ($\phi^I$ l1 #:t t) ($\phi^I$ l2 #:t t) ($\phi^I$ l3 #:t t))))
    \end{lstlisting} & [] & [] \\ \hline
    $\longrightarrow_{\beta p}$&\begin{lstlisting}[mathescape]
(program ([l2 (linklet () (a) (define-values (a) 10) (void))]
           [l3 (linklet () (k) (k))])
  (let-inst t (make-instance)
     (seq ($\phi^I$ (L$\alpha$ () ((Export a1 a a) (Export k1 k k))
                   (define-values (k) (lambda () (var-ref a1)))
                   (var-set! k1 k) (void)) #:t t)
           ($\phi^I$ l2 #:t t)
           ($\phi^I$ l3 #:t t))))
    \end{lstlisting} & [] & [] \\ \hline
    $\longrightarrow_{\beta p}$&\begin{lstlisting}[mathescape]
(program ([l3 (linklet () (k) (k))])
  (let-inst t (make-instance)
     (seq ($\phi^I$ (L$\alpha$ () ((Export a1 a a) (Export k1 k k))
                   (define-values (k) (lambda () (var-ref a1)))
                   (var-set! k1 k) 1) #:t t)
           ($\phi^I$ (L$\alpha$ () ((Export a1 a a))
                   (define-values (a) 10) (var-set! a1 a) (void)) #:t t)
           ($\phi^I$ l3 #:t t))))
    \end{lstlisting} & [] & [] \\ \hline
    $\longrightarrow_{\beta p}$&\begin{lstlisting}[mathescape]
(program ()
  (let-inst t (make-instance)
     (seq ($\phi^I$ (L$\alpha$ () ((Export a1 a a) (Export k1 k k))
                   (define-values (k) (lambda () (var-ref a1)))
                   (var-set! k1 k) (void)) #:t t)
           ($\phi^I$ (L$\alpha$ () ((Export a1 a a))
                   (define-values (a) 10) (var-set! a1 a) (void)) #:t t)
           ($\phi^I$ (L$\alpha$ () ((Export k1 k k)) ((var-ref k1))) #:t t))))
    \end{lstlisting} & [] & [] \\ \hline
  \end{tabular}
\end{table}

After compiling and loading the linklets, we start by creating and
binding the target that will act like the \emph{top-level} target
instance as described above.


\begin{table}[h!]
  \centering
  \footnotesize
  \begin{tabular}{lc|c|c}
    &\textbf{program} & \textbf{$\rho$} & \textbf{$\sigma$} \\ \hline \hline
    $\longrightarrow_{\beta p}$&\begin{lstlisting}[mathescape]
(program ()
  (let-inst t ((void) li)
     (seq ($\phi^I$ (L$\alpha$ () ((Export a1 a a) (Export k1 k k))
                   (define-values (k) (lambda () (var-ref a1)))
                   (var-set! k1 k) (void)) #:t t)
           ($\phi^I$ (L$\alpha$ () ((Export a1 a a))
                   (define-values (a) 10) (var-set! a1 a) (void)) #:t t)
           ($\phi^I$ (L$\alpha$ () ((Export k1 k k)) ((var-ref k1))) #:t t))))
    \end{lstlisting} & [] & [li $\rightarrow$ (linklet-instance)] \\ \hline
    $\longrightarrow_{\beta p}$&\begin{lstlisting}[mathescape]
(program ()
  (seq ($\phi^I$ (L$\alpha$ () ((Export a1 a a) (Export k1 k k))
                (define-values (k) (lambda () (var-ref a1)))
                (var-set! k1 k) (void)) #:t t)
        ($\phi^I$ (L$\alpha$ () ((Export a1 a a))
                (define-values (a) 10) (var-set! a1 a) (void)) #:t t)
        ($\phi^I$ (L$\alpha$ () ((Export k1 k k)) ((var-ref k1))) #:t t))))
    \end{lstlisting} & [] & \thead{[t $\rightarrow$ (linklet-instance), \\li $\rightarrow$ (linklet-instance)]} \\ \hline
  \end{tabular}
\end{table}

Now we start the instantiation of the linklets.

\begin{table}[h!]
  \centering
  \footnotesize
  \begin{tabular}{lc|c|c}
    &\textbf{program} & \textbf{$\rho$} & \textbf{$\sigma$} \\ \hline \hline
    $\longrightarrow_{\beta p}$&\begin{lstlisting}[mathescape]
(program ()
  (seq ($\phi^I$ (L$\beta$ t (define-values (k) (lambda () (var-ref a1)))
                      (var-set! k1 k) (void)))
        ($\phi^I$ (L$\alpha$ () ((Export a1 a a))
                (define-values (a) 10) (var-set! a1 a) (void)) #:t t)
        ($\phi^I$ (L$\alpha$ () ((Export k1 k k)) ((var-ref k1))) #:t t))))
    \end{lstlisting} & \thead{[k1 $\rightarrow$ $var_k$, \\a1 $\rightarrow$ $var_a$]} & \thead{[$var_a$ $\rightarrow$ uninit,\\ $var_k$ $\rightarrow$ uninit,\\ t $\rightarrow$ (linklet-instance \\ (a $var_a$) (k $var_k$)), \\li $\rightarrow$ (linklet-instance)]} \\ \hline
    $\longrightarrow_{\beta r}$&\begin{lstlisting}[mathescape]
(program ()
  (seq ($\phi^I$ (L$\beta$ t (define-values (k)
                         (closure () (var-ref a1) ((k1 $var_k$) (a1 $var_a$))))
                      (var-set! k1 k) (void)))
        ($\phi^I$ (L$\alpha$ () ((Export a1 a a))
                (define-values (a) 10) (var-set! a1 a) (void)) #:t t)
        ($\phi^I$ (L$\alpha$ () ((Export k1 k k)) ((var-ref k1))) #:t t))))
    \end{lstlisting} & \thead{[k1 $\rightarrow$ $var_k$, \\a1 $\rightarrow$ $var_a$]} & \thead{[$var_a$ $\rightarrow$ uninit,\\ $var_k$ $\rightarrow$ uninit,\\ t $\rightarrow$ (linklet-instance \\ (a $var_a$) (k $var_k$)), \\li $\rightarrow$ (linklet-instance)]} \\ \hline
    $\longrightarrow_{\beta p}$&\begin{lstlisting}[mathescape]
(program ()
  (seq ($\phi^I$ (L$\beta$ t (var-set! k1 k) (void)))
        ($\phi^I$ (L$\alpha$ () ((Export a1 a a))
                (define-values (a) 10) (var-set! a1 a) (void)) #:t t)
        ($\phi^I$ (L$\alpha$ () ((Export k1 k k)) ((var-ref k1))) #:t t))))
    \end{lstlisting} & \thead{[k $\rightarrow$ $cell_1$,\\k1 $\rightarrow$ $var_k$,\\a1 $\rightarrow$ $var_a$]} & \thead{[$cell_1$ $\rightarrow$ closure,\\$var_a$ $\rightarrow$ uninit,\\$var_k$ $\rightarrow$ uninit,\\t $\rightarrow$ (linklet-instance\\(a $var_a$) (k $var_k$)),\\li $\rightarrow$ (linklet-instance)]} \\ \hline
    $\longrightarrow_{\beta r}$&\begin{lstlisting}[mathescape]
(program ()
  (seq ($\phi^I$ (L$\beta$ t (var-set! k1 (closure () (var-ref a1) ((k1 $var_k$) (a1 $var_a$))))
                       (void)))
        ($\phi^I$ (L$\alpha$ () ((Export a1 a a))
                (define-values (a) 10) (var-set! a1 a) (void)) #:t t)
        ($\phi^I$ (L$\alpha$ () ((Export k1 k k)) ((var-ref k1))) #:t t))))
    \end{lstlisting} & \thead{[k $\rightarrow$ $cell_1$,\\k1 $\rightarrow$ $var_k$,\\a1 $\rightarrow$ $var_a$]} & \thead{[$cell_1$ $\rightarrow$ closure,\\$var_a$ $\rightarrow$ uninit,\\$var_k$ $\rightarrow$ uninit,\\t $\rightarrow$ (linklet-instance\\(a $var_a$) (k $var_k$)),\\li $\rightarrow$ (linklet-instance)]} \\ \hline
    $\longrightarrow_{\beta r}$&\begin{lstlisting}[mathescape]
(program ()
  (seq ($\phi^I$ (L$\beta$ t (void)))
        ($\phi^I$ (L$\alpha$ () ((Export a1 a a))
                (define-values (a) 10) (var-set! a1 a) (void)) #:t t)
        ($\phi^I$ (L$\alpha$ () ((Export k1 k k)) ((var-ref k1))) #:t t))))
    \end{lstlisting} & \thead{[k $\rightarrow$ $cell_1$,\\k1 $\rightarrow$ $var_k$,\\a1 $\rightarrow$ $var_a$]} & \thead{[$cell_1$ $\rightarrow$ closure,\\$var_a$ $\rightarrow$ uninit,\\$var_k$ $\rightarrow$ closure,\\t $\rightarrow$ (linklet-instance\\(a $var_a$) (k $var_k$)),\\li $\rightarrow$ (linklet-instance)]} \\ \hline
    $\longrightarrow_{\beta p}$&\begin{lstlisting}[mathescape]
(program ()
  (seq ((void) t)
        ($\phi^I$ (L$\alpha$ () ((Export a1 a a))
                (define-values (a) 10) (var-set! a1 a) (void)) #:t t)
        ($\phi^I$ (L$\alpha$ () ((Export k1 k k)) ((var-ref k1))) #:t t))))
    \end{lstlisting} & \thead{[k $\rightarrow$ $cell_1$,\\k1 $\rightarrow$ $var_k$,\\a1 $\rightarrow$ $var_a$]} & \thead{[$cell_1$ $\rightarrow$ closure,\\$var_a$ $\rightarrow$ uninit,\\$var_k$ $\rightarrow$ closure,\\t $\rightarrow$ (linklet-instance\\(a $var_a$) (k $var_k$)),\\li $\rightarrow$ (linklet-instance)]} \\ \hline
  \end{tabular}
\end{table}
