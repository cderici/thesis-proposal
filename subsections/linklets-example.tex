\subsubsection{Extended Example: Top-level REPL via Linklets}
\label{subsec:toplevel-example}

This section presents an extended example that demonstrates
\textbf{i)} how linklets are evaluated, and \textbf{ii)} how the
top-level interaction is implemented through the linklet
variables. First we will see a general overview of the mechanism, then
we will state the example in the formalism that's shown in
\secref{subsec:linklet-semantics} to see the essential details.

The top-level itself is quite complicated and beyond the scope of our
study. The example here aims to demonstrate the essential logic behind
the linklet variables that implement the functionalities like the
top-level repl\footnote{Read-Eval-Print Loop}.

The implementation of the top-level in the expander provides a
specially handled linklet instance, namely the \emph{top-level} that
contains a reference to the namespace and all the defined variables,
acting like a top-level environment. Each expression entered in the
repl is evaluated by inserting in a linklet that imports the necessary
bindings for expansion bookkeeping, compiling and instantiating with
the \emph{top-level} as the target instance.

Let's consider the following
interactions in \figref{fig:toplevel-interaction}.

\begin{wrapfigure}{r}{0.3\textwidth}
  \begin{mdframed}
    \begin{verbatim}
> (define k (lambda () a))
> (define a 10)
> (k)
10
\end{verbatim}
    \caption{Top-level Example}
    \label{fig:toplevel-interaction}

  \end{mdframed}
\end{wrapfigure}

asdasd
