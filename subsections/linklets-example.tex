\subsubsection{Extended Example: Top-level REPL via Linklets}
\label{subsec:toplevel-example}

This section presents an extended example that demonstrates
\textbf{i)} how linklets are evaluated, and \textbf{ii)} how the
top-level interaction is implemented through the linklet
variables. First we will see a general overview of the mechanism, then
we will state the example in the formalism that's shown in
\secref{subsec:linklet-semantics} to see the essential details.

\begin{wrapfigure}{r}{0.3\textwidth}
  \vspace{-0.5cm}
  \begin{mdframed}
    \begin{verbatim}
> (define k (lambda () a))
> (define a 10)
> (k)
10
\end{verbatim}
    \caption{Top-level Example}
    \label{fig:toplevel-interaction}
  \end{mdframed}
\end{wrapfigure}

The top-level itself is quite complicated and beyond the scope of our
study. The example here aims to demonstrate the essential logic behind
the linklet variables that implement the functionalities like the
top-level repl\footnote{Read-Eval-Print Loop}.

The implementation of the top-level in the expander provides a
specially handled linklet instance, namely the \emph{top-level} that
contains a reference to the namespace and all the defined variables,
acting like a top-level environment. Each expression entered in the
repl is evaluated by inserting in a linklet that imports the necessary
bindings for expansion bookkeeping, compiling and instantiating with
the \emph{top-level} as the target instance. For the clarity here we
simplify the generated linklets and only show the essential parts.

As our example, let's consider the interactions in
\figref{fig:toplevel-interaction}. The key point here is that the
variable "a" is defined \emph{after} it is used within the function
"k". Let's see what happens at each step:

\paragraph{$>$ (define k (lambda () a))}

As we briefly explained earlier, this expression is put into a linklet
body, then the linklet is instantiated over the \emph{top-level}
instance. Recall that a linklet instance is a finite mapping from
symbols to variables. Initially our target instance is empty.

SOMEHOW SHOW THE LINKLET AND THE INSTANCE BEFORE AND AFTER THE INSTANTIATE

(linklet (omitted) (a k) (define-values (k) (lambda () a)))

compiled

(linklet (omitted) ((a0 a) (k0 k)) (define-values (k) (lambda () (variable-ref (LinkletVar a0)))) (variable-set! (LinkletVar k0) k \#f))
(target)

instantiate

(target (a a0) (k k0))

The most essential point in this example is the handling of the
undefined identifier "a". When the expander generates a linklet
containing an unidentified identifier, it puts the identifier among
the exported variables of the linklet (despite having no corresponding
definition for it), allowing it to be handled via the linklet
variables. During the compilation of the linklet, a new linklet
variable is created for it with the special value
\emph{uninitialized}, and when the binding for the identifier is
needed, it will be dynamically provided through the target,
essentially forming a low-level dynamic scope. Note that we have the
variable mappings for both "a" and the "k" within the target instance
after the instantiation.

\paragraph{$>$ (define a 10)}

SOMEHOW SHOW THE LINKLET AND THE INSTANCE BEFORE AND AFTER THE INSTANTIATE

(linklet (omitted) (a) (define-values (a) 10))

compiled

(linklet (omitted) ((a1 a)) (define-values (a) 10) (variable-set! (LinkletVar a1) a #f))
(target (a a0) (k k0))

\paragraph{$>$ (k)}

SOMEHOW SHOW THE LINKLET AND THE INSTANCE BEFORE AND AFTER THE INSTANTIATE

(linklet (k) (k))

compiled

(linklet module ((k1 k)) ((variable-ref (LinkletVar k1))))
(target (a a0) (k k0))
