%\subsection{Meta-tracing \& Self Hosting}
\label{subsec:self-hosting-problem}

\vspace{-0.4cm}

\paragraph{Meta-tracing \& Self Hosting} With Racket being easier to host in different VMs and already having
an interpreter for it in Pycket, it is logical to consider
self-hosting Racket on RPython, thereby turning Pycket into an
implementation of Racket. However, self-hosting on a meta-tracing
framework poses challanges in terms of performance. The most common
problem is the inconsistency in the performance improvements,
especially for programs that has frequent changes in the control
flow. In 2011, Mozilla reported having this problem in their
TraceMonkey JIT where they observed a massive speed-up in performance
for the parts of the programs that has tight\footnote{having only a
  single exit/loop point} loops, while suffering substantially in
performance for other parts, stating ``You lose more when slow than
you gain when fast'' \cite{mozblog}. This problem becomes more
apparent when meta-tracing is involved, because the program that is
being traced is another interpreter, which inherently involves many
different control flow paths at each iteration in its dispatch loop
\cite{bolz15-meta-vm}. Increasing the meta level of the VM by tracing
a self-hosting interpreter, therefore, poses even a greater
challenge. Moreover, self-hosting a modern functional programing
language like Racket requires additional functionalities that
naturally entail dispatch-loops, such as expander, reader, regular
expressions and fast-load serialization (FASL).
