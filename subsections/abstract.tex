This thesis presents the challenge of having an efficient
implementation of a self-hosting functional language on a meta-tracing
framework and argues that it is achievable.

I will start by discussing how meta-tracing enables rapid prototyping
efficient dynamic language implementations. We will discuss the use of
interpreter hints in meta-tracing to allow discovering and tracing the
loops in a user program, rather than in the interpeter evaluating
it. I will also discuss how meta-tracing is shown to make a fast
Racket interpreter, namely Pycket. This will be the backbone of our
further discussion.

Next I will introduce the linklets to study how Racket made its
implementation more independent from the run-time, thereby achieving a
greater flexibility to target different virtual machines starting with
Chez Scheme. I will develop a formal model of linklets to investigate
how they work, and study how the communication between the front and
back-end of Racket is established via the linklets.

Then I will demonstrate the process of incorporating linklets into the
Pycket to turn it into a full implementation of Racket. We will
discuss how the self-hosting of Racket works with bootstrapping
linklets, and demonstrate how loading and evaluating a Racket module
on Pycket works from start to end.

Next I will provide some preliminary experiments to reveal the key
performance issues fundamental to implementing a self-hosting
functional language on a meta-tracing framework, such as the GC
pressure in the run-time and tracing programs with non-trivial control
flow paths.

Finally I will dicuss approaches to addressing those issues and
propose solutions with additional experiments and formalism that will
improve the performance enough to justify using meta-tracing to
implement VMs for self-hosting functional languages.
