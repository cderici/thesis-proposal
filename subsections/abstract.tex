This thesis presents solution approaches for some of the fundamental
performance issues of a self-hosting functional language
implementation on a meta-tracing framework and argues that a
competitive performance is achievable.

I will start by discussing how meta-tracing enables rapid prototyping
of efficient dynamic language implementations. To establish the
backbone of our further discussion, we will discuss the use of
interpreter hints in meta-tracing to allow discovering and tracing the
loops in a user program, rather than in the interpeter evaluating
it. I will also discuss how meta-tracing makes a fast Racket
interpreter, namely Pycket.

Next I will introduce the linklet form to study how Racket made its
implementation more independent from the run-time, thereby achieving a
greater flexibility to target different virtual machines. I will
develop a formal model of linklets to investigate how they work, and
study how the communication between the front and back-end of Racket
is established via the linklets.

Then I will demonstrate the process of incorporating linklets into the
Pycket to eventually turn it into an implementation of the
self-hosting Racket. We will discuss how Racket's self-hosting works
with bootstrapping linklets, how loading and evaluating a Racket
module works on Pycket from start to end, and demonstrate with
examples how functionalities like top-level repl implemented with
linklets

Next I will study the performance issues that are fundamental to
implementing a self-hosting interpreter on a meta-tracing framework. I
will provide preliminary experiments and discussion to reveal the key
points of these issues that we have observed during the improvement of
Pycket.

Finally I will discuss approaches to addressing the issues we
introduced earlier and propose solutions with additional experiments
and formalism, as well as demonstration on the new Pycket, to improve
the performance enough to justify using meta-tracing to implement VMs
for self-hosting functional languages.
