This thesis presents the challenge of having an efficient
implementation of a self-hosting functional language on a meta-tracing
framework and argues that it is achievable.

I start by discussing how meta-tracing enables rapid prototyping
efficient dynamic language implementations, and how it is shown to
make a fast Racket interpreter, namely Pycket.

Next I will study how Racket made its implementation a bit more
independent from the run-time using linklets, thereby achieving a
greater flexibility to target different virtual machines.

Then I will demonstrate the process of incorporating linklets into
Pycket to turn it into a full implementation of Racket on RPython.

Then I will study the performance issues fundamental to
meta-tracing an interpreter for a self-hosting language using Pycket
as an example.

Finally I will dicuss approaches to addressing those issues and
propose solutions.
