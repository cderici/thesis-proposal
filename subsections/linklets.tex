\subsection{Language Portability via Linklets}
\label{subsec:linklets}

\begin{itemize}
\item Here's how linklets work.

\item Here's how Racket achieved greated portability with linklets.
\end{itemize}


Linklets are lambda-like forms that consume and produce potentially
mutable variables (instead of values). They act as units of
compilation in Racket runtime, where for a given Racket module the
expander produces a bundle of linklets that will be consumed by the
runtime compiler. The following sections explain what linklets are and
how they work to evaluate Racket modules and implement functionalities
such as top-level.

Figure \ref{fig:linklet-source} shows the grammar of the linklet
language, which is a minimal subset of the Fully Expanded Racket
Programs extended with the \verb|linklet| form and toplevel
definitions. Figure \ref{fig:racket-source} presents the grammar for
the Racket part of the language, which is used within the linklet
body. In the remainder of this document we will call this sub-language
Racket Core (RC). RC is basically a $\lambda$-calculus with a handful of
syntactic extensions, as well as some additional forms to manipulate
linklet variables (e.g. \emph{var-ref}).

A linklet consists of a set of variable definitions and expressions,
an exported subset of the defined variable names, a set of variables
to export from the linklet despite having no corresponding definition,
and a set of imports that provide other variables for the linklet to
use. To run a linklet, it is instantiated as a \emph{linklet
  instance}. When a linklet is instantiated, it receives other linklet
instances for its imports, and it extracts a specified set of
variables that are exported from each of the given instances. The
newly created linklet instance provides its exported variables for use
by other linklets. Moreover, instead of freshly creating an instance,
an instantiation may be provided with an additional linklet instance
as an input, namely the target instance. In this case, the target
instance is used and modified for the linklet definitions and
expressions during the evaluation of the body expressions, and the
result is the value of the last expression in the linklet. In the
remainder of this document we will use the names ``targeted
instantiation'' and ``regular instantiation'' for these two modes.


\begin{figure}[tbp]
  \begin{align*}
    L  &::= (\textbf{linklet}\:((imp\: \ldots)\: \ldots)\:(exp\: \ldots)\:l\dash top\:\ldots\:e)\\
    l\dash top &::= (\textbf{define\dash values}\:(x)\:e)\:|\:e \\
    imp &::= x\:|\:(x x) \tag*{[external internal]}\\
    exp &::= x\:|\:(x x) \tag*{[internal external]}
  \end{align*}
  \caption{Linklet Source Language}
  \label{fig:linklet-source}
\end{figure}

\begin{figure}[tbp]
  \begin{align*}
    e &::=\; x\; |\; v\; |\; (e\; e\; \ldots)\; |\; (\textbf{if}\; e\; e\; e)\; |\; (o\; e\; e)\; \\
    \; &|\; (\textbf{begin}\; e\; e\; \ldots)\; |\; (\textbf{lambda}\; (x_{\_!\_}\; \ldots)\; e)\\
    \; &|\; (\textbf{set!}\; x\; e)\; |\; (\textbf{raises}\; e)\\
    \; &|\; (\textbf{var\dash ref}\; x)\; |\; (\textbf{var\dash ref/no\dash check}\; x) \\
    \; &|\; (\textbf{var\dash set!}\; x\; e)\; |\; (\textbf{var\dash set/check\dash undef!}\; x\; e) \\
    v   &::=\; n\; |\; b\; |\; c\; |\; (void)\;|\; uninit \\ %;; values
    c   &::=\; (closure\; (x\; \ldots)\; e\; \rho) \\
    n   &::=\; number \\
    b   &::=\; true\; |\; false \\
    var &::=\; variable\dash not\dash otherwise\dash mentioned \\ %;; variables
    x,\; cell &::=\; variable\dash not\dash otherwise\dash mentioned \\ %;; variables
    o  &::=\; +\; |\; * |\; < \\
    \rho   &::=\; ((x\; var)\; \ldots) \tag*{[Environment]}\\
    \sigma   &::=\; ((var\; v)\; \ldots) \tag*{[Store]}\\
    E   &::=\; []\; |\; (v\; \ldots\; E\; e\; \ldots)\; |\; (o\; E\; e)\; |\; (o\; v\; E) \\
    \;  &|\; (\textbf{var\dash set!}\; x\; E)\; |\; (\textbf{var\dash set/check\dash undef!}\; x\; E) \\
    \;  &|\; (\textbf{begin}\; v\; \ldots\; E\; e\; \ldots)\; |\; (\textbf{set!}\; x\; E)\; |\; (\textbf{if}\; E\; e\; e)
  \end{align*}
  \caption{Racket Core Language (RC)}
  \label{fig:racket-source}
\end{figure}


The minimal API for a runtime to use the linklets consists of the
functions \textbf{i)} \verb|compile-linklet| for preparing a linklet
for instantiation and \textbf{ii)} \verb|instantiate-linklet| for
running a compiled linklet.

\def\la {\bm{L_\alpha}}
\def\lb {\bm{L_\beta}}

\begin{figure}[tbp]
  \begin{align*}
    CL             &::=\; \bm{\Phi^C}(L)\\
    L\dash obj     &::= (\la\; c\dash imps \; c\dash exps \; l\dash top \ldots)\\
    \;           &\;|\; (\lb\; x\; l\dash top\; \ldots) \\
    LI             &::=\; (\textbf{linklet\dash instance}\; (x\; var)\; \ldots) \\
    c\dash imps    &::=\; ((imp\dash obj\; \ldots)\; \ldots)\\
    c\dash exps    &::=\; (exp\dash obj\; \ldots)\\
    imp\dash obj   &::=\; (\textbf{Import}\; x\; x\; x) \tag*{[id int ext]} \\
    exp\dash obj   &::=\; (\textbf{Export}\; x\; x\; x) \tag*{[id int ext]} \\
  \end{align*}
  \hfill \footnotesize $\Phi^C : \textbf{compile\dash linklet}$
\caption{Linklet Runtime Language}
\label{fig:linklet-runtime}
\end{figure}

Figure \ref{fig:linklet-runtime} shows the grammar of the run\dash
time language for linklets, i.e. expressions that represent the
run\dash time objects used in the instantiation of linklets. Given a
linklet L, \verb|compile-linklet| produces a linklet object $\la$ that
is ready to be instantiated. $\la$ consists of \textit{Import} and
\textit{Export} objects for the identifiers denoting the imported and
exported variables, and the body expressions with potentially some
additional expressions for variable manipulations in the run\dash
time. Each \textit{Import} and \textit{Export} object consists of two
identifiers that are internal and external with respect to the linklet
body (a variable may be imported/exported with different names), and
one gensymed\footnote{is there a good name for this?} identifier for
internal use during the instantiation. After compiling, before an
actual instantiation of a linklet object begins, the \textit{Import}
and \textit{Export} objects are processed. As we detail in Section
\ref{sec:instantiate}, references for the imported variables are
collected from the given import instances, and additional variables
may be created for exported variables. $\lb$ denotes a compiled
linklet object after such a preparation step, with which the actual
instantition may begin. Note that $\lb$ contains only a reference (to
a linklet instance that is used during the instantiation) and the body
expressions, and doesn't contain the import and export objects, as the
instantiation uses the variables that are already prepared for
them. Finally a \textbf{linklet\dash instance} is represented as a
finite mapping from identifiers to variables (i.e. cells in the heap).


\begin{figure}[h]
\begin{verbatim}
(program
  ([l (linklet () (a) (define-values (a) 5) a)])
  (let-inst i (instantiate-linklet l)
    (instance-variable-value i a)))
\end{verbatim}
\caption{An example \textit{program}}
\label{fig:program-example}
\end{figure}

Since the linklets are lambda\dash like binding forms and not capable
of starting a computation by themselves, we introduce a top\dash level
form, namely \verb|program|. A \verb|program| consists of a set of
given linklets to be loaded and a single top\dash level expression to
perform a computation. Figure \ref{fig:linklet-program} shows the
grammar for \verb|program|s. We provided the \verb|let-inst| form to
name a linklet instance, and \verb|seq| form to perform multiple
computations in sequence. Figure \ref{fig:program-example} shows an
example \verb|program|.

\begin{figure}[h!]
  \begin{align*}
    p           &::=\; (\textbf{program}\; ((x\; L)\; \ldots)\; p\dash top) \\
    p\dash top  &::=\; v\; |\; (\textbf{let\dash inst}\; x\; p\dash top\;p\dash top)\; |\; (\textbf{seq}\; p\dash top\; \ldots) \\
    \;          &\;|\; \bm{\Phi^I}(l\dash ref\; x\; \ldots)\; |\; \bm{\Phi^I}(l\dash ref\; x\; \ldots\; \textbf{\#:t}\; x)\; |\; \bm{\Phi^V}(x\; x) \\
    l\dash ref &::=\; x\; |\; L\dash obj \\
    v           &::=\; ....\;|\;(v\;x) \\
    EP          &::=\; [] \\
    \;          &\;|\; \bm{\Phi^I}((\lb\; x\; v\; \ldots\; EP\; l\dash top\; \ldots)\; x\; \ldots) \\
    \;          &\;|\; (\textbf{define\dash values}\:(x)\:EP) \\
    \;          &\;|\; (\textbf{let\dash inst}\; x\; EP\; p\dash top)\; |\; (\textbf{seq}\; v\; \ldots\; EP\; p\dash top\; \ldots) \\
    \;          &\;|\; (\textbf{program}\; ()\; EP) \\
  \end{align*}
  \hfill \footnotesize $\Phi^V : \textbf{instance\dash variable\dash value}\:,\: \Phi^I : \textbf{instantiate\dash linklet}$
\caption{Linklet Programs}
\label{fig:linklet-program}
\end{figure}

\paragraph{Program Evaluation} starts with compiling and loading the given
set of linklets. After the given linklets are loaded, the expression
in the body is evaluated. Figure \ref{fig:reduction} shows the
reduction relation, $\beta_p$, that implements the evaluation, together
with the reduction relation for RC, $\beta_r$, that is shown in Figure
\ref{fig:rc-reduction}. For clarity, the reduction of programs and
linklet instantiation is defined together by the same relation ($\beta_p$)
so that the whole evaluation is defined as a relexive-transitive
closure of the same relation, generating a single small-step sequence
of reductions from programs to values. The next subsections present
the compilation and instantiation of linklets, as well as two example
reductions for each instantiation modes (i.e. regular and targeted).


Before a linklet is ready for instantiation, several passes over the
linklet are performed to ensure a correct information flow between the
variables in the run\dash time. Figure \ref{fig:compile} shows the
entire compilation procedure.

For each import and export identifier (and possibly their
corresponding renamings), \textit{Import} and \textit{Export} objects
are created with the gensymed\footnote{[BETTER NAME]} references that
will be used to name the variables that are dynamically created during
the instantiation. Additionally, a pass on the linklet body collects
all the identifiers that are defined within the linklet body (via
\verb|define-values|) and all the mutated identifiers (i.e. targets of
\verb|set!| expressions).

Given a set of \textit{Import}s ($\mathit{C_I}$) and a set of
\textit{Export}s ($\mathit{C_E}$), a set of toplevel defined
identifiers ($\mathit{X_T}$) and a set of mutated identifiers
($\mathit{X_M}$), the compilation starts to process each expression in
the linklet body one by one, performing the following actions
depending on the type of expression:

\begin{itemize}
 \item For a `(\textbf{define-values} (x) e)` expression where \newline
   $(\textbf{Export}\; x_{gen}\; x\; x_{ext}) \in \mathit{C_E}$ (i.e. x
   is exported), add an additional `(\textbf{var-set!} $x_{gen}$ x)`
   expression.
 \item For a `(\textbf{set!} x e)` expression where \newline
   $(\textbf{Export}\; x_{gen}\; x\; x_{ext}) \in \mathit{C_E}$ (i.e. x
   is exported), turn the expression into
   `(\textbf{var-set/check-undef!} $x_{gen}$ e)`.
 \item For any identifier x where $(\textbf{Import}\; x_{gen}\; x\;
   x_{ext}) \in \mathit{C_I}$ (i.e. x is imported), turn the identifier
   into \newline `(\textbf{var-ref/no-check} $x_{gen}$)` form.
  \item For any identifier x where $(\textbf{Export}\; x_{gen}\; x\; x_{ext}) \in
    \mathit{C_E}$ and either $x \in \mathit{X_M}$ or $x \notin \mathit{X_T}$,
    turn the identifier into `(\textbf{var-ref} $x_{gen}$)` form.
  \item For any other expression e, recurse if e has subforms, return
    otherwise.
\end{itemize}

\begin{figure}[h]
\begin{verbatim}
(compile-linklet
  (linklet ((a)) (x)
    (define-values (x) 5)
    (set! x 6)
    (+ x a)))

(Lα (((Import a1 a a))) ((Export x1 x x))
  (define-values (x) 5)
  (var-set! x1 x)
  (var-set/check-undef! x1 6)
  (+ (var-ref x1) (var-ref/no-check a1)))
\end{verbatim}
\caption{An example linklet compilation [TODO: add line numbers]}
\label{fig:compile-example}
\end{figure}

Figure \ref{fig:compile-example} shows an example compilation. The
`(\textbf{var-set!} x1 x)` expression is added since the defined
identifier x is exported. The `(\textbf{var-set/check-undef!} x1 6)`
is introduced in place of the `(\textbf{set!} x 6)` expression, as the
target identifier x is exported. Also, in `(+ x a)` expression,
`(\textbf{var-ref} x1)` and `(\textbf{var-ref/no-check} a1)`
expressions are introduced for the exported identifier x and imported
identifier a respectively.


After the compilation and loading phase in the program evaluation, all
the linklet objects are in $\la$ form and ready to be instantiated. In
general, an instantiation consists of preparing the import and export
variables and then evaluating the expressions in the linklet body,
much like $\beta$-reducing a lambda application.

Recall that a linklet instantiation has two modes; \textbf{i)} regular
instantiation, where a new instance is being created and returned
after evaluating the body, and \textbf{ii)} targeted instantiation,
where the given target instance is used during the
instantiation. Since the only difference between the two modes is the
return value based on whether a target instance is provided as an
input, for simplicity in the model we remove the distinction between
the modes by turning the regular instantiation into a targeted
instantiation with a fresh empty instance. Moreover, the instantiation
returns a pair of the last expression's value and the target instance
being used. The caller of the \emph{instantiate-linklet} therefore
decides to use either the value or the instance.

Imported variables are collected from the given linklet instances by
simply going through each set of \emph{Import} objects in the $\la$
and the corresponding linklet instance that provides the
variables. The collected variable references are then put into the
environment with the names in the corresponding \emph{Import} objects
that were generated for the variables during the
compilation. Figure~\ref{fig:inst-utils} shows the processing of both
the imported and exported variables.

The use of the target instance during an instantiation comes into play
when the exported variables are being processed. Recall that a linklet
may export a variable despite having no corresponding definition for
it. This is achieved by for each \emph{Export} object in $\la$
grabbing the variable from the target instance if it has a mapping for
it and making it available in the environment. Otherwise a fresh
variable is created and its mapping is placed into the environment and
into the target instance as well.

After processing the import and export variables, the $\la$ object is
reduced into $\lb$, which triggers the evaluation of the expressions
in the linklet body. The linklet compilation makes sure that every
variable reference within the body uses the appropriate variable, and
at the beginning of the instantiation the preparation of the import
and export variables explained above provides the actual variables in
the environment to be modified and used. The reduction then proceeds
with the evaluation of the body expressions one-by-one. After all the
expressions are processed, a pair of the last expression's value and
the reference to the target instance is returned.

\newcommand{\redinput}[1]{EP\;\llbracket #1 \rrbracket, \rho, \sigma\;}
\newcommand{\redoutput}[3]{EP\;\llbracket #1 \rrbracket, #2, #3\;}

\newcommand{\rcinput}[1]{EP\;\llbracket E\;\llbracket #1 \rrbracket \rrbracket, \rho, \sigma\;}
\newcommand{\rcoutput}[3]{EP\;\llbracket E\;\llbracket #1 \rrbracket \rrbracket, #2, #3\;}
\def\rcrel {&\longrightarrow_{\beta r}\;}

\def\where {\textbf{where}\;}
\def\rel {&\longrightarrow_{\beta p}\;}

\begin{figure*}[tbp]
  \begin{align*}
    \rcinput{x} \rcrel \rcoutput{\sigma[\rho[x]]}{\rho}{\sigma} \\
    \rcinput{(\textbf{var\dash ref}\; x))} \rcrel \rcoutput{v}{\rho}{\sigma}\; \where x \in \rho, \rho[x] \in \sigma, v = \sigma[\rho[x]]\; \&\; v \neq uninit\\
    \rcinput{(\textbf{var\dash ref/no\dash check}\; x))} \rcrel \rcoutput{v}{\rho}{\sigma}\; \where v = \sigma[\rho[x]] \\
    \rcinput{(\textbf{var\dash set!}\; x\; v)} \rcrel \rcoutput{(void)}{\rho}{\sigma_1} \where \sigma_1 = \sigma[\rho[x]\rightarrow v]\\
    \rcinput{(\textbf{var\dash set/check\dash undef!}\; x\; v)} \rcrel \rcoutput{(void)}{\rho}{\sigma_1} \where \sigma_1 = \sigma[\rho[x]\rightarrow v], x \in \rho\\
    \rcinput{(\textbf{set!}\; x\; v)} \rcrel \rcoutput{(void)}{\rho}{\sigma_1} \where \sigma_1 = \sigma[\rho[x]\rightarrow v]\\
    \rcinput{(\textbf{lambda}\; (x\; \ldots)\; e)} \rcrel \rcoutput{(closure\; (x\; \ldots)\; e\; \rho)}{\rho}{\sigma} \\
    \rcinput{(\textbf{begin}\; v\; \ldots\; v_l)} \rcrel \rcoutput{v_l}{\rho}{\sigma} \\
    \rcinput{(\textbf{if}\; v\; e_1\; e_2)} \rcrel \rcoutput{e}{\rho}{\sigma} \where e = \begin{cases*}
      e_2 & if $v = false$ \\
      e_1 & otherwise
    \end{cases*}\\
    \rcinput{(o\; v_1\; v_2)} \rcrel \rcoutput{\delta(v_1,v_2)}{\rho}{\sigma} \\
    \rcinput{((closure\; (x\; \ldots)\; e\; \rho_c)\;v\;\ldots)} \rcrel \rcoutput{e}{\rho_1}{\sigma_1} \where \rho_1 = \rho_c\overline{[x\rightarrow cell]}, \sigma_1 = \sigma\overline{[cell\rightarrow v]}, cell \notin \sigma \\
  \end{align*}
  \caption{Reduction Relation for Racket Core}
  \label{fig:rc-reduction}
\end{figure*}

\begin{figure*}[tbp]
  \begin{align*}
    \redinput{\bm{\Phi^V}(x_{li},\; x)} \rel \redoutput{v}{\rho}{\sigma} \where v\; =\; \sigma[(\sigma[x_{li}])[x]] \\
    \redinput{(\textbf{let\dash inst}\; x\; (v\;x_i)\; p\dash top)} \rel \redoutput{p\dash top}{\rho}{\sigma_1} \where \sigma_1 = \sigma[x\rightarrow \sigma[x_i]] \\
    \redinput{(\textbf{seq}\; v_{1}\; \ldots\; v_{n})} \rel \redoutput{v_{n}}{\rho}{\sigma} \\
    \redinput{(\bm{\Phi^I}\; (\lb\; x\; v\; \ldots\; v_{l}))} \rel \redoutput{(v_{l}\; x)}{\rho}{\sigma} \\
    EP\;\llbracket (\bm{\Phi^I}(\lb\; x_{t}\; v_p\; \ldots\; (\textbf{define\dash values}\; (x)\; v)\; l\dash top\; \ldots)) \rrbracket, \rel \redoutput{\bm{\Phi^I}(\lb\; x_{t}\; v_p\; \ldots\; l\dash top\; \ldots)}{\rho_1}{\sigma_1} \\
    \;          &\;\; \where \rho_1 = \rho[x\rightarrow cell],\; \sigma_1 = \sigma[cell\rightarrow v] \\
    \redinput{(\bm{\Phi^I}(\la\; c\dash imps \; c\dash exps \; l\dash top \ldots)\; LI\; \ldots)} \rel \redoutput{(\bm{\Phi^I}(\la\; c\dash imps \; c\dash exps \; l\dash top\; \ldots)\; LI\; \ldots\; \textbf{\#:t}\; x_{t})}{\rho}{\sigma_1} \\
    \;          &\;\; \where x_t \notin \mathit{dom}(\sigma), \sigma_1 = \sigma[x_t\rightarrow (linklet\dash instance)] \\
    EP\;\llbracket (\bm{\Phi^I}(\la\; c\dash imps \; c\dash exps \; l\dash top \ldots)\; LI\; \ldots\; \textbf{\#:t}\; x_{t}) \rrbracket, \rho, \sigma\; \rel \redoutput{(\bm{\Phi^I}(\lb\; x_{t}\; l\dash top\; \ldots))}{\rho_2}{\sigma_1} \\
    \; &\;\; \where \rho_1 = \mathit{V}^I(c\dash imps, (LI\; \ldots), \rho) \\
    \;          &\;\; \hspace{0.4cm} (\rho_2, \sigma_1) = \mathit{V}^E(c\dash exps, x_t, \rho_1, \sigma) \\
    (\textbf{program}\; ((x\;L),(x_1\;L_1)\;\ldots)\;p\dash top), \rho, \sigma &\longrightarrow_{\beta p}\; (\textbf{program}\; ((x_1\;L_1)\;\ldots)\;p\dash top[x := \bm{\Phi^C}(L)]), \rho, \sigma \\ \\
    eval^{\longrightarrow_{\beta p}}(p) &=\; v\; \textbf{if}\; p, (), () \twoheadrightarrow_{\beta p\cup\beta r} \; (\textbf{program}\; ()\;(v\;\_)) \\
  \end{align*}
  \hfill \footnotesize $\Phi^V : \textbf{instance\dash variable\dash value}\:,\: \Phi^I : \textbf{instantiate\dash linklet},\: \mathit{V}^I : \textbf{get  import variables},\: \mathit{V}^E : \textbf{create variables for exports}$
\caption{Reduction Relation}
\label{fig:reduction}
\end{figure*}

\newcommand{\runio}[1]{run\dash prog(#1, \rho, \sigma)\;}
\newcommand{\prog}[2]{(\textbf{program}\; (\textbf{use\dash linklets}\;#1 )\;#2)}
\def\transrel {&\twoheadrightarrow_{p}\;}

\begin{figure*}[tbp]
  \begin{align*}
    \mathit{V}^I : c\dash imps \times (LI\;\ldots) \times \rho &\longrightarrow \rho \\
    \mathit{V}^I((\overline{(\overline{(\textbf{Import}\; x_{id}\;x_{int}\;x_{ext})})_n}), (\overline{LI_n}), \rho) &= \rho\overline{[x_{id}\rightarrow LI_n[x_{ext}]]}\\ \\
    %% \mathit{V}^I : c\dash imps \times (LI\;\ldots) \times \rho &\longrightarrow \rho \\
    %% \mathit{V}^I((), (), \rho) &= \rho \\
    %% \mathit{V}^I((()\;(imp\dash obj\; \ldots)\;\ldots), (LI\;LI_r\;\ldots), \rho) &= \mathit{V}^I(((imp\dash obj\; \ldots)\;\ldots), (LI_r\;\ldots), \rho) \\
    %% \mathit{V}^I((((\textbf{Import}\; x_{id}\;x_{int}\;x_{ext})\;imp\dash obj\;\ldots)\;\ldots), (LI\;LI_r\;\ldots), \rho) &= \mathit{V}^I(((imp\dash obj\;\ldots)\;\ldots), (LI\;LI_r\;\ldots), \rho[x_{id}\rightarrow var]) \\
    %% \;          &\; \where var = LI(x_{ext}) \\ \\
    \mathit{V}^E : c\dash exps \times x \times \rho \times \sigma &\longrightarrow \rho \times \sigma \\
    \mathit{V}^E((exp\dash obj\;\ldots), x_t, \rho, \sigma) &= \mathit{P}(exp\dash obj, x_t, \sigma[x_t], \rho, \sigma)\;\ldots\\
    \mathit{P}((\textbf{Export}\;x_{id}\;x_{int}\;x_{ext}), x_t, LI_t, \rho, \sigma) &= \begin{cases*}
      \rho_1, \sigma & if $\;x_{ext} \in dom(LI_t)$, \where $\rho_1 = \rho[x_{id}\rightarrow LI_t[x_{ext}]]$ \\
      \rho_1, \sigma_1 & if $\;x_{ext} \notin dom(LI_t)$, \where $\rho_1 = \rho[x_{id}\rightarrow var_{new}]$\\
      \;        &\; \hspace{2.6cm} $var_{new} \notin \sigma$ \\
      \;        &\; \hspace{2.6cm} $\sigma_1 = \sigma[var_{new}\rightarrow uninit,$ \\
      \;        &\; \hspace{3.2cm} $x_t\rightarrow LI_t[x_{ext}\rightarrow var_{new}]]$
    \end{cases*} \\
    %% \mathit{P}((\textbf{Export}\;x_{id}\;x_{int}\;x_{ext}), x_t, \rho, \sigma) &= \rho_1, \sigma_1\; \where \\
    %% \;          &\; \rho_1 = \rho[x_{id}\rightarrow A] \\
    %% \;          &\; A = \begin{cases*}
    %%   LI_t[x_{ext}] & if $x_{ext} \in LI_t$ \\
    %%   var_{new}     & o/w
    %% \end{cases*} \\
    %% \;          &\; \sigma_1 = \begin{cases*}
    %%   \sigma & if $x_{ext} \in LI_t$ \\
    %%   \sigma[var_{new}\rightarrow uninit, x_t\rightarrow LI_t[x_{ext}\rightarrow var_{new}]] & o/w
    %% \end{cases*} \\
    %% \;          &\; var_{new} \notin LI_t \cup \sigma \\
    %% \;          &\; LI_t = \sigma[x_t] \\
  \end{align*}
  \hfill \footnotesize $\mathit{V}^I : \textbf{get  import variables},\: \mathit{V}^E : \textbf{create variables for exports}$
\caption{Instantiation Utilities}
\label{fig:inst-utils}
\end{figure*}


\begin{figure*}[tbp]
  \small
  \begin{align*}
    \Phi^C : L &\longrightarrow L\dash obj \\
    \Phi^C((\textbf{linklet}\; ((imp\; \ldots)\; \ldots)\; (exp\; \ldots)\; l\dash top\; \ldots))\; &=\; (\la\; c\dash imps\; c\dash exps\; l\dash top_1\; \ldots) \\
    \;        &\; \where l\dash top_1\; \ldots\; =\; \Phi^C_{\textbf{e}}(l\dash top,(),(x_M\;\ldots),(x_T\;\ldots))\; \ldots \\
    \;        &\; c\dash imps = \mathit{create\dash import\dash objects}(((imp\; \ldots)\; \ldots)) \\
    \;        &\; c\dash exps = \mathit{create\dash export\dash objects}((exp\; \ldots)) \\
    \;        &\; (x_T\;\ldots) = toplevel\dash ids(l\dash top\; \ldots) \\
    \;        &\; (x_M\;\ldots) = mutated\dash ids(l\dash top\; \ldots) \\ \\
    \mathit{create\dash import\dash objects} : ((imp\; \ldots)\; \ldots) &\longrightarrow ((imp\dash obj\; \ldots)\; \ldots) \\
    \mathit{create\dash import\dash objects}((imp_n\; \ldots)\; \ldots)\; &=\; ((imp\dash obj_n\; \ldots)\; \ldots) \\
    \;   \where imp\dash obj_n  &=\; \begin{cases*}
      (\textbf{Import}\;x_{gen}\;x\;x) & if $imp_n = x$ \\
      (\textbf{Import}\;x_{gen}\;x_{int}\;x_{ext}) & if $imp_n = (x_{ext}\;x_{int})$ \\
    \end{cases*} \\
    \;     x_{gen} &\notin\; ((imp_n\; \ldots)\; \ldots) \\ \\
    \mathit{create\dash export\dash objects} : (exp\; \ldots) &\longrightarrow (exp\dash obj\; \ldots) \\
    \mathit{create\dash export\dash objects}(exp_n\; \ldots)\; &=\; (exp\dash obj\; \ldots) \\
    \;   \where exp\dash obj_n  &=\; \begin{cases*}
      (\textbf{Export}\;x_{gen}\;x\;x) & if $exp_n = x$ \\
      (\textbf{Export}\;x_{gen}\;x_{int}\;x_{ext}) & if $exp_n = (x_{int}\;x_{ext})$ \\
    \end{cases*} \\
    \;     x_{gen} &\notin\; (exp_n\; \ldots) \\ \\
    \Phi^C_{\textbf{e}} : l\dash top \times \mathit{C_I} \times \mathit{C_E} \times \mathit{X_L} \times \mathit{X_M} \times\mathit{X_T} &\longrightarrow l\dash top\; \ldots \\
    \;          &\;\; \where \mathit{C_I} : (c\dash imps\;\ldots), \mathit{C_E} : (c\dash exps\;\ldots) \\
    \;          &\;\; \mathit{X_L} : (x\;\ldots), \mathit{X_M} : (x\;\ldots), \mathit{X_T} : (x\;\ldots) \\
    \Phi^C_{\textbf{e}}((\textbf{define\dash values}\:(x)\:e),\mathit{C_I}, \mathit{C_E}, \mathit{X_L},\mathit{X_M},\mathit{X_T})\; &=\; \begin{cases*}
      (\textbf{define\dash values}\:(x)\:e), (var\dash set!\; x_{gen}\; x)\;\; \;\; if\; (\textbf{Export}\; x_{gen}\; x\; x_{ext}) \in\;\mathit{C_E} \\
      (\textbf{define\dash values}\:(x)\:e) \hspace{2.5cm} \;\; otherwise
    \end{cases*} \\
    \Phi^C_{\textbf{e}}((\textbf{set!}\; x\; e),\mathit{C_I}, \mathit{C_E}, \mathit{X_L},\mathit{X_M},\mathit{X_T})\; &=\; \begin{cases*}
      (\textbf{var\dash set/check\dash undef!}\; x_{gen}\; e) \hspace{1.4cm} if\; (\textbf{Export}\; x_{gen}\; x\; x_{ext}) \in\;\mathit{C_E}\\
      (\textbf{set!}\; x\; e) \hspace{4.3cm} otherwise
    \end{cases*} \\
    \Phi^C_{\textbf{e}}(x,\mathit{C_I}, \mathit{C_E}, \mathit{X_L},\mathit{X_M},\mathit{X_T})\; &=\; \begin{cases*}
      (\textbf{var\dash ref/no\dash check}\; x_{gen}) \hspace{1.5cm} if\; (\textbf{Import}\; x_{gen}\; x\; x_{ext}) \in\;\mathit{C_I} \\
      (\textbf{var\dash ref}\; x_{gen}) \hspace{2.9cm} if\; (\textbf{Export}\; x_{gen}\; x\; x_{ext}) \in\;\mathit{C_E}\;\; \&\;\; x\in \mathit{X_M} \\
      (\textbf{var\dash ref}\; x_{gen}) \hspace{2.9cm} if\; (\textbf{Export}\; x_{gen}\; x\; x_{ext}) \in\;\mathit{C_E}\;\; \&\;\; x\notin \mathit{X_T} \\
      x \hspace{4.6cm} otherwise
    \end{cases*} \\
    \Phi^C_{\textbf{e}}(e,\mathit{C_I}, \mathit{C_E}, \mathit{X_L},\mathit{X_M},\mathit{X_T})\; &=\; recurse \\
  \end{align*} \\
\caption{Compile Linklet}
\label{fig:compile}
\end{figure*}
