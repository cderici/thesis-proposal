\subsection{Meta-tracing}

Meta-tracing \textit{just-in-time (JIT)} compilers allow programming
language developers and researchers to realize an efficient
implementation of a programming language in a short period of
time. Given only the interpreter of the language (possibly annotated
with some hints), meta-tracing can not only automatically produce an
efficient executable for the interpreter (by compiling it to C), but
also build a JIT compiler into the binary as well. For example, the
PyPy, a Python interpreter built on the RPython meta-tracing framework
achieved a speedup by a factor of 6.54 over the interpreter (in C),
outperforming even the CPython itself. \cite{bolz09} Another example
of this is Pycket, a high-level interpreter for Racket based on the
CEK abstract machine, again, built on the RPtyhon framework. Pycket
was able to outperform Racket's own JIT and other highly-optimized
Scheme compilers, while also demonstrating a significant reduction on
the overhead of contracts and gradual typing. \cite{pycket15}
