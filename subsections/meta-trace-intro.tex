%\subsection{Meta-tracing \& Pycket}
\label{subsec:meta-trace}

\vspace{-0.2cm}

\paragraph{Meta-tracing \& Pycket} Over the past several decades the tracing \emph{just-in-time (JIT)}
virtual machines (VM) gained a significant popularity for their
substantial performance benefits. On another front, instead of tracing
a program, tracing an interpreter evaluating that program, namely the
meta-tracing, allowed developers and researchers to rapidly prototype
fast VMs for dynamic languages \cite{bolz09}. Given only the
interpreter of the language (possibly annotated with some hints),
meta-tracing can automatically produce an efficient executable for the
interpreter (by effectively compiling the interpreter to C), and also
build a JIT compiler into the binary as well. For example, the PyPy, a
Python interpreter built via the RPython meta-tracing framework has
been a big success, achieved a speedup by a factor of 6.54 over the
interpreter (in C), outperforming even the CPython itself
\cite{bolz09}. In 2014, Bolz and collaborators discovered that
meta-tracing can increase the performance of dynamic functional
programming languages as well \cite{bolz14-racket}, and created
Pycket, a high-level interpreter for Racket based on the control,
environment, and continuation (CEK) abstract machine
\cite{felleisen87}. Built on the RPtyhon framework, Pycket was able to
outperform Racket's own VM and other highly-optimized Scheme
compilers, while also demonstrating a significant reduction on the
overhead of contracts and gradual typing \cite{pycket15,pycket17}.
