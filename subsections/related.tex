%\newpage

\section{Related Work}
\label{sec:related}

\paragraph{Higher-order dynamic VMs}
Dynamic VM implementations are becoming increasingly popular for their
rapid prototyping potential, as the effort required to implement a new
VM from scratch is often quite large. Therefore, instead of manually
implementing a VM in a low-level language such as C, it is often
argued that building on top of an existing object-oriented
general-purpose VMs or dynamic integration via generating a VM using a
"specification" of a language allows easier and more maintainable
implementations with competitive performance
\cite{bolz_how_no:07}. One of the major actors for the former approach
is the \emph{GraalVM}, which is a modified version of the Java HotSpot
VM on the JVM (Java Virtual Machine). GraalVM uses a language
implementation framework called Truffle, and a method-based JIT
compiler called Graal to implement VMs on the JVM for dynamic
languages such as Javascript, Ruby and Python \cite{graal:13}. As
opposed to building on top of VM, the \emph{RPython} project
introduced the idea of automatically generating a VM from a language
specification represented as an interpreter via meta-tracing
\cite{rpython07}. For example, as mentioned in the introduction, PyPy
is an implementation of Python that is built on the RPython
meta-tracing framework that generated a VM including a tracing JIT for
Python with a better performance than the CPython itself. In addition
to the RPython framework, there are other meta-tracing systems as
well, such as the SPUR, a tracing JIT compiler for CIL bytecode
\cite{miller2004common}, meta-tracing languages that are implemented
in C\# \cite{spur}.

\paragraph{Tracing JIT VMs}
Like PyPy, Pycket is built on the RPython meta-tracing framework as
well, and this study tries to identify approaches on achieving
efficient self-hosting on Pycket running a tracing JIT. Therefore it
would be relevant to mention here some notable tracing JIT VMs
too. Initially introduced by the Dynamo project \cite{dynamo},
trace-based compilation is successfully utilized by many VMs including
some commercial VMs such as Mozilla's TraceMonkey JavaScript VM
\cite{tracemonkey}, Adobe's Tamarin ActionScript VM \cite{tamarin}, as
well as some research VMs such as LuaJIT \cite{luajit:08}, Converge
\cite{converge:05}, Lambdamachine \cite{lambdamachine} and PyHaskell
\cite{pyhaskell}.

\paragraph{Optimizing VM performance in a meta-tracing context}
As we mentioned in \secref{subsec:rpython}, there are a lot of dynamic
optimizations performed by the RPython back-end that the language
interpreter should be in sync with. There are two main approaches for
a VM author to optimize such a system: \textbf{i)} improve the
interpreter performance, \textbf{ii)} modify the interpreter to
produce traces that are easier to optimize by JIT. Both of these
approaches operate on the interpreter, therefore the improvements
should focus on the common patterns that are specific to the language
being interpreted. For example, along with the general JIT
improvements such as value promotion and edlibale functions, PyPy and
Converge both focus on optimizing the objects, classes and modules
\cite{bolz15-meta-vm}.

\paragraph{Self-hosting}
Similarly, Bolz and contributors state that in general-purpose VMs,
the compiler and the language being implemented should be semantically
in sync to achieve a good performance \cite{bolz_how_no:07,
  runtime-feedback:11}. As we discussed in
\secref{subsec:performance}, we study the relationship between the
semantics of the programs common to self-hosting and the interpreter
\& trace performance. Unfortunately, to the best of our knowledge, no
studies of self-hosting on meta-tracing has been done so far. However,
there are self-hosting VMs for dynamic languages, such as the Tachyon
VM for JavaScript, which is a bootstrapping JIT compiler that is
completely hand-coded in JavaScript and doesn't use any automatic
generation techniques \cite{self-hostJSvm:11}. Although Tachyon is
quite different than Pycket and bootstraps the VM itself, it uses
control-flow graphs in SSA form as an intermediate representation
(IR), similar to Pycket. Moreover, it uses interesting techniques that
might prove useful for Pycket as well, such as conditional constant
propogation via abstract interpretation \cite{sccp:91}. Additionally,
there are low-level techniques to deal with control-flow issues on VMs
caused by large amount of branches, such as indirect branch prediction
\cite{branch-predict:03}, which might be adapted to be useful in
tracing as well. However, in our case implementing such techniques
would require modifying the underlying RPython framework, where our
approach is at the language (interpreter) level.

\paragraph{Managing abstraction levels}
A big part of the problems we tackle in this thesis is caused by the
fact that adding self-hosting on Pycket adds another level of
abstraction that needs to be semantically in-sync with the run-time
meta-tracing VM. There has been related research studying to reduce
complexity in systems involving multiple levels of abstractions, such
as type-directed partial evaluation \cite{tdpe:99} and multi-stage
programming \cite{multi-stage:97, multi-stage:17}. Amin and Rompf
studied how to collapse a tower of interpreters interpreting
each-other into a compiler to remove the interpretation overhead
\cite{collapse:17}. They developed a multi-level lambda calculus and a
meta-circular evaluator, namely Pink, to demonstrate the collapse of
arbitrarily many levels of self-interpretation.

Another work that is relevant to our investigation is optimizing
hierarchically layered VMs using trace compilation, researched by
Yermolovich and contributors \cite{layering:09}. Similar to the
RPython framework, they propose to run a guest VM (Lua VM
\cite{luajit:08}) on top of a host VM with a trace-based JIT compiler
(Tamarin \cite{tamarin}). Along with optimizing the guest VM, the host
VM also allows the guest VM to provide hints about its interpreter
loop that can specify which parts of the guest VM should not be traced
for its own, thereby taking into account the guest VMs workload as
well.

\paragraph{Space concerns \& heap allocated continuations}
Many studies investigate the space complexity and optimizations aiming
to increase performance. Some use \emph{repurposed} JIT compilers
(RJIT) to benefit from the optimizations that the dynamic languages
enable, such simplification of control-flow graphs
\cite{dynStatComp:12} and run-time type feedbacks \cite{stJITdyn:12}
in the context of adding support for dynamic languages to an existing
static-language VMs. Others use higher-level techniques such as
hidden-classes in Pycket \cite{pycket17}, similar to the maps
introduced by Self \cite{self-maps:89} and used by PyPy
\cite{runtime-feedback:11}.

As we discussed in the performance section, one of the big concerns we
identified in self-hosting is the garbage collection overhead caused
by the long-chains of heap allocated continuations. Allocating
activation records on the heap is studied extensively in the context
of compiling with \cite{comp-cont:07} or without
\cite{comp-without-cont:17} continuations \cite{whatever:19,
  compWithContLLVM:16}. With that said, perhaps the most relevant work
to our investigation with the stackful and CEK interpreters working
together is by Hieb and contributors \cite{cont-heap-stack:90}, where
they provide an approach to allocate activation records on the stack
that doesn't require the stack to be copied when a continuation is
created, thereby establishing an upper bound on the amount of copying
which is otherwise unbounded. They also provide techniques for stack
overflow/underflow recovery, which is isomorphic to continuation
creation and re-instantiation.

To the best of our knowledge, no comprehensive synthesis of issues and
opportunities of self-hosting on a meta-tracing VM has been
done. Therefore a head-to-head comparison of Pycket with such a system
is impossible. However, existing Racket implementations such as
Racket's previous (generic JIT-based) and current (Chez
\cite{racket-on-chez-19}) run-time are suitable for performance
comparison with Pycket hosting Racket.



% What's diff about TC from the related work
% small summary for diesel....
% - diesel supports x
%  - calculus supports some subset of x
% we support y, which covers most of x but also foo
