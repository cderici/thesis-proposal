%\newpage

\section{Related Work}
\label{sec:related}

\paragraph{Higher-order dynamic VMs}
Dynamic VM implementations are becoming increasingly popular for their
rapid prototyping potential, as the effort required to implement a new
VM from scratch is often quite large. Therefore, instead of manually
implementing a VM in a low-level language such as C, it is often
argued that building on top of an existing object-oriented
general-purpose VMs or dynamic integration via generating a VM using a
"specification" of a language allows easier and more maintainable
implementations with competitive performance
\cite{bolz_how_no:07}. One of the major actors for the former approach
is the \emph{GraalVM}, which is a modified version of the Java HotSpot
VM on the JVM (Java Virtual Machine). GraalVM uses a language
implementation framework called Truffle, and a method-based JIT
compiler called Graal to implement VMs on the JVM for dynamic
languages such as Javascript, Ruby and Python \cite{graal:13}. As
opposed to building on top of VM, the \emph{RPython} project
introduced the idea of automatically generating a VM from a language
specification represented as an interpreter via meta-tracing
\cite{rpython07}. For example, as mentioned in the introduction, PyPy
is an implementation of Python that is built on the RPython
meta-tracing framework that generated a VM including a tracing JIT for
Python with a better performance than the CPython itself.


% a head-to-head comparison difficult or impossible.

%% To the best of our knowledge, no comprehensive synthesis of issues and
%% opportunities has been done.


\cite{bolz15-meta-vm} stated that in general-purpose object-oriented
VMs, the compiler is optimized for the language, or group of
languages, it was designed for. If a language’s semantics are
significantly different and thus do not fit the VM well, it will
perform poorly— despite a highly optimized underlying VM.

\cite{ast:12} argues that AST interpreters allow for much more
extensive modifications.

\cite{branch-predict:03} optimizing indirect branch prediction
accuracy in VM interpreters

Virtual machine showdown: Stack versus registers.

Another approach is to add support for dynamic languages to an
existing high-performance static-language VM \cite{stJITdyn:12,
  dynStatComp:12}.

heap allocated continuations \cite{whatever:19, compWithContLLVM:16}

collapsing towers of interpreters \cite{collapse:17}

Hierarchical Layering of VMs \cite{layering:09}

Bootstrapping a Self-hosted Research Virtual Machine for JavaScript:
An Experience Report \cite{self-hostJSvm:11}

% since we are not able to

%% Bilmemne et al presents ..... In contrast to Pycket, in this system
%% ..... However, Pycket does not .....

%% \paragraph{just description}
%% Occurrence Typing Occurrence typing [43, 42] extends the type system
%% with a proposition environment that represents the information on the
%% types of bindings down conditional branches. These propositions are
%% then used to update the types associated with bindings in the type
%% environment down branches so binding occurrences are given different
%% types depending on the branches they appear in, and the conditionals
%% that lead to that branch.

% Notably, they further perform a qualitative analysis aiming to
% identify the reasons why

% There are other works that are relevant to our investigations of ....

% one sentence

% What's diff about TC from the related work
% small summary for diesel....
% - diesel supports x
%  - calculus supports some subset of x
% we support y, which covers most of x but also foo
