\subsubsection{Top-level REPL via Linklets}
\label{subsec:toplevel-example}

Let's take a look at how a VM that implements linklets can easily
implement functionalities such as the Racket top-level
repl\footnote{read-eval-print loop} by loading the expander linklet
into its run-time. Note that the top-level itself is quite complicated
and beyond the scope of our study. The example here aims to provide a
general overview of the interaction between the run-time and the
functionalities provided by the language through the linklets, for
this exercise in particular the expander linklet. Recall that the
expander implements functions such as read, expand, eval, and outputs
a bundle of linklets for a given Racket code.

First let's briefly describe what linklets are and how linklets
operate. In \secref{subsec:linklets-semantics} I will provide linklets'
operational semantics with a formalism in more detail.

A linklet consists of a set of variable definitions and expressions,
an exported subset of the defined variable names, a set of variables
to export from the linklet despite having no corresponding definition,
and a set of imports that provide other variables for the linklet to
use. Some linklet examples can be seen in
\figref{fig:racket-expand-example}. To run a linklet, it is
instantiated as a \emph{linklet instance}, which is a collection of
linklet variables. When a linklet is instantiated, it receives other
linklet instances for its imports, and it extracts a specified set of
variables that are exported from each of the given instances. The
newly created linklet instance provides its exported variables for use
by other linklets. Moreover, an instantiation may be given an
additional linklet instance as an input, namely the target
instance. In this case, the variables in the target instance is used
and modified for the linklet definitions and expressions during the
evaluation of the body expressions, and the result is the value of the
last expression in the linklet (instead of an instance). In the
remainder of this document we will use the names ``targeted
instantiation'' and ``regular instantiation'' for these two modes.

The minimal API for a virtual machine to implement the linklets
consists of the functions \textbf{i)} \racketcode{compile-linklet} for
preparing a linklet for instantiation and \textbf{ii)}
\racketcode{instantiate-linklet} for running a compiled linklet. In
particular, the VM uses the \racketcode{compile-linklet} to take an
s-expression for a linklet as input and produce a linklet object in a
representation of its choice, containing the run-time AST for the
codes inside the linklet. Therefore, loading a linklet into its
run-time for a VM means that the functions provided by the linklet are
\emph{callable} during the run-time. For example after loading the
expander linklet, a VM can \emph{call} the functions from Racket's
expander such as read, expand, eval to use Racket's module system,
macro system etc.

\begin{wrapfigure}[9]{r}{0.3\textwidth}
  \vspace{-0.6cm}
  \begin{mdframed}
    \begin{verbatim}
> (define k (lambda () a))
> (define a 10)
> (k)
10
\end{verbatim}
    \caption{Top-level Example}
    \label{fig:toplevel-interaction}
  \end{mdframed}
\end{wrapfigure}

As our example, let's consider the interactions in
\figref{fig:toplevel-interaction}. The key point here is that the
variable "a" is defined \emph{after} it is used within the function
"k". To implement this repl, the VM will keep a dedicated linklet
instance that we will refer to as the \emph{top-level}
instance. Recall that a linklet instance is a collection of linklet
variables. In our study we assume that it is implemented as a finite
mapping of identifiers to linklet variables. Initially the top-level
instance will be an empty instance.

The key point is the interaction between the language and the run-time
through the expander linklet. At each step in the repl, the VM gets a
Racket expression to evaluate. Then it applies the \racketcode{read}
and \racketcode{expand} functions imported from the expander linklet,
on the given expression. For clarity, we will omit the non-essential
details and focus only on the simplified run-time linklet for the
expression.

Let's see what happens at each interaction:

\paragraph{$>$ (define k (lambda () a))}

As we briefly explained earlier, the VM will call the expander's
\racketcode{expand} function, which will output a linklet containing (the
expanded version of) this expression. Then the VM will use
\racketcode{instantiate-linklet} to instantiate it over the \emph{top-level}
instance, which is initially empty.

\begin{figure}[h!]
  \small
  \begin{subfigure}[b]{0.25\textwidth}
    \begin{mdframed}
\begin{verbatim}
(linklet (...) (a k)
  (define-values (k)
    (lambda () a)))
\end{verbatim}
    \end{mdframed}
    \caption{Linklet \\ generated by \texttt{expand}}
    \label{fig:1}
  \end{subfigure}
  %
  \begin{subfigure}[b]{0.38\textwidth}
    \begin{mdframed}
\begin{verbatim}
(linklet (...) ((a0 a) (k0 k))
  (define-values (k)
    (lambda ()
      (variable-ref (LinkletVar a0))))
  (variable-set! (LinkletVar k0) k))
\end{verbatim}
    \end{mdframed}
    \caption{Linklet \\ compiled by VM}
    \label{fig:2}
  \end{subfigure} \hfill
    %
  \begin{subfigure}[b]{0.15\textwidth}
    \begin{mdframed}
      \begin{align*}
        a_0 &\rightarrow var_a \\
        k_0 &\rightarrow var_k \\
      \end{align*}
    \end{mdframed}
    \caption{Environment \\ after compile}
    \label{fig:2}
  \end{subfigure}
  %
  \begin{subfigure}[b]{0.15\textwidth}
    \begin{mdframed}
      \begin{align*}
        a &\rightarrow var_a \\
        k &\rightarrow var_k \\
      \end{align*}
    \end{mdframed}
    \caption{Target \\ after instantiate}
    \label{fig:2}
  \end{subfigure}
\end{figure}

The important point about the top-level part in this example is the
handling of the undefined identifier "a". When the expander generates
a linklet containing an unidentified identifier, it puts the
identifier among the exported variables of the linklet (despite having
no corresponding definition for it), allowing it to be handled via the
linklet variables. During the compilation of the linklet on the VM, a
new linklet variable is created for it with the special value
\emph{uninitialized}, and when the binding for the identifier is
needed, it will be dynamically provided using the target instance (in
this case using the \emph{top-level} instance), essentially forming a
low-level dynamic scope. Note that we have the variable mappings for
both "a" and the "k" within the target instance after the
instantiation, containing the values \emph{uninitialized} and a
closure with no arguments and body with a reference to $var_a$
respectively.

\paragraph{$>$ (define a 10)}

When this new expression is inserted in the repl, the VM calls the
\racketcode{expand} again to produce a linklet, then it compiles and
instantiates it over the same \emph{top-level} target instance.

%\vspace{-0.5cm}

\begin{figure}[h!]
  \small
  \begin{subfigure}[b]{0.26\textwidth}
    \begin{mdframed}
\begin{verbatim}
(linklet (...) (a)
  (define-values (a) 10))
\end{verbatim}
    \end{mdframed}
    \caption{Linklet \\ generated by \texttt{expand}}
    \label{fig:1}
  \end{subfigure}
  %
  \begin{subfigure}[b]{0.38\textwidth}
    \begin{mdframed}
\begin{verbatim}
(linklet (...) ((a1 a))
  (define-values (a) 10)
  (variable-set! (LinkletVar a1) a))
\end{verbatim}
    \end{mdframed}
    \caption{Linklet \\ compiled by VM}
    \label{fig:2}
  \end{subfigure} \hfill
  \begin{subfigure}[b]{0.15\textwidth}
    \begin{mdframed}
      \begin{align*}
        a_1 &\rightarrow var_a \\
      \end{align*}
    \end{mdframed}
    \caption{Environment \\ after compile}
    \label{fig:2}
  \end{subfigure}
  \begin{subfigure}[b]{0.15\textwidth}
    \begin{mdframed}
      \begin{align*}
        a &\rightarrow var_a \\
        k &\rightarrow var_k \\
      \end{align*}
    \end{mdframed}
    \caption{Target \\ after instantiate}
    \label{fig:2}
  \end{subfigure}
\end{figure}

Note that the \racketcode{compile-linklet} inserted the \emph{variable-set!}
expression in the body to set the value of the variable bound by the
external name "a". During the instantiation, this variable exists in
two possible ways. When the exported variables are processed at the
start of the instantiation, if the target instance doesn't have a
variable mapping for "a", then a new variable will be created with the
\emph{uninitialized} value (as in the previous step), otherwise if the
target does have a mapping for "a" (which it does in this case), then
the target's variable will be used during the instantiation, and will
be set by \emph{variable-set!}. Note that because of this the binding
for $a_1$ in the environment above is set to be the $var_a$ (instead
of a new variable), since the target does has a mapping for "a".

\paragraph{$>$ (k)}

At this step, we have everything we need to evaluate this expression
thanks to the \emph{top-level} target instance.

\begin{figure}[h!]
  \small
  \begin{subfigure}[b]{0.25\textwidth}
    \begin{mdframed}
\begin{verbatim}
(linklet (...) (k) (k))
\end{verbatim}
    \end{mdframed}
    \caption{Linklet \\ generated by \texttt{expand}}
    \label{fig:1}
  \end{subfigure}
  %
  \begin{subfigure}[b]{0.38\textwidth}
    \begin{mdframed}
\begin{verbatim}
(linklet (...) ((k1 k))
  ((variable-ref (LinkletVar k1))))
\end{verbatim}
    \end{mdframed}
    \caption{Linklet \\ compiled by VM}
    \label{fig:2}
  \end{subfigure} \hfill
    %
  \begin{subfigure}[b]{0.15\textwidth}
    \begin{mdframed}
      \begin{align*}
        k_1 &\rightarrow var_k \\
      \end{align*}
    \end{mdframed}
    \caption{Environment \\ after compile}
    \label{fig:2}
  \end{subfigure}
  %
  \begin{subfigure}[b]{0.15\textwidth}
    \begin{mdframed}
      \begin{align*}
        a &\rightarrow var_a \\
        k &\rightarrow var_k \\
      \end{align*}
    \end{mdframed}
    \caption{Target \\ after instantiate}
    \label{fig:2}
  \end{subfigure}
\end{figure}

When instantiating the compiled linklet, because of the first step,
the value for the variable $var_k$ is resolved as a closure with no
arguments and containing a reference to $var_a$ in the body. Similarly
when that closure is applied, the value for the $var_a$ is resolved as
10, thanks to the second step.
