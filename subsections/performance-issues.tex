\subsection{Performance Issues with Self-Hosting on JIT Compilers}
\label{subsec:performance}

With many micro-benchmarks and larger experiments, the existing Pycket
implementation is shown to be significantly performant in evaluating
Racket code. It benefits from the the existing generic optimizations
in the RPython framework, including common subexpression elimination,
copy propogation, constant folding, loop invariant code motion,
malloc-removal and the inlining that naturally comes for free from
tracing \cite{loop-aware:12, hotpath:06, malloc-removal:11}, as well
as from many improvements on the interpreter such as environment
pruning, data structure specialization, strategies, as well as some
gradual typing related improvements such the use of hidden-classes
\cite{pycket15, pycket17}.

Adding the linklets layer to the front-end of the existing
implementation doesn't have any effect on the benchmark performance,
since the linklets layer only adds the capability of processing and
running linklets and doesn't include or change anything in the
Pycket's back-end. However, having the linklets allows Pycket to run
the programs that are much larger than micro-benchmarks or individual
experiment programs. To give a perspective, the offline generated
expander linklet that is processed in Pycket at boot itself is roughly
5MB file containing the s-expressions of 2642 functions. Another
example is, to run a single Racket program written in the
\emph{racket/base} language (such as repl for instance), more than 100
Racket modules need to be expanded and instantiated first just to load
the language before running the program itself.

In addition to the problem of being have to evaluate larger programs,
self-hosting also reveals an issue that is fundamental to the tracing
JIT compilers, which is being have to evaluate large dispatch-loops
that create complex control flows that are significantly hard for
tracing. In this section we study both of these problems and attempt
to understand the key parts to address them in the next section.

\subsubsection{Garbage Collection (GC) Pressure}

As briefly described in \secref{subsec:rpython}, the flow graphs and
the C code that are generated during the translation in the RPython
framework assume automatic memory management. Therefore the produced
program is linked with a GC that is implemented within the framework
(in RPython), by inspecting all the graphs and turning all
\texttt{malloc} operations into calls to the GC. RPython's garbage
collector, namely the \emph{minimark GC} is a three-generations
semispace copying generational collector \cite{pypy06, bolz:14,
  gc:16}.

\begin{itemize}
\item GC pressure on the heap... minimark GC (current) is three semispace generational GC (uses nursery for the young objects, mark \& sweep for olds)
  works well when
  - objects that die young...
  - most of the collected objs are newly created post the prev gc cycle


  the depth of real world Racket programs like
  read, expand etc doesn't play well with this

\item (main issue) Confused traces because of complex control flow
  (branchy dispatch loops)

  branchy code, with experiments, show traces

\end{itemize}
