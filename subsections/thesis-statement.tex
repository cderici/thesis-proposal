My thesis statement is:

\begin{quote}
  An efficient self-hosting functional programming language on a
  meta-tracing JIT is achievable.
\end{quote}

I will support this thesis statement with the following research
contributions:

\begin{itemize}
  \item \textit{An implementation of self-hosting Racket on RPython.}
    I will demonstrate that Pycket has all the necessary run-time
    support (e.g. primitives, data structures, error handling etc.)
    to load and evaluate Racket code without requiring any external
    support.
  \item \textit{Approaches in achieving good performance} I will
    identify the fundamental issues in self-hosting a language such as
    Racket on a meta-tracing JIT compiler, and provide approaches in
    addressing them. I will vet the run-time performance of the
    proposed approaches on the improved Pycket and argue why these
    approaches enable a full and efficient implementation.

    %% propose, demonstrate, evalauate and discuss some solution
    %% approaches that address the numerous performance issues that are
    %% fundamental to self-hosting a language such as Racket on a
    %% meta-tracing JIT compiler, such as garbage collection pressure on
    %% the heap caused by deep continuation chains on the interpreter,
    %% and the problem with tracing loops with complex control flow paths
    %% as discused above.



    %% Preliminary analyses indicate, however, there might be a
    %% more general problem that subsumes the aformentioned issues, which
    %% is that the amount of information that's been communicated between
    %% the interpreter and the run-time might not be not enough, for
    %% example for the JIT to maneuver around such a dispatch loop
    %% exercising a large number of branches.


\end{itemize}

%% I will augment this thesis statement with one of the following research directions:

%% \begin{enumerate}
%%   \item this can be done
%% \end{enumerate}
