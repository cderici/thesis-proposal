\section{Thesis Statement}

My thesis statement is:

\begin{quote}
  A full and efficient implementation of a self-hosting functional
  programming language is possible with a meta-tracing JIT.
\end{quote}

I will support this thesis statement with the following:

\begin{itemize}
  \item \textit{A full implementation of self-hosting Racket on
    RPython.} I will demonstrate that Pycket has all the necessary
    runtime support (e.g. primitives, data structures, error handling
    etc.)  that Racket code assumes and is able to load and evaluate
    any Racket code, thereby being able to also run and pass Racket's
    own test suite, consisting of more than 800.000 test cases (modulo
    the ones that are acceptable corner cases such as error-message
    differences etc.).
  \item \textit{An efficient implementation of self-hosting Racket on
    RPython.} I will demonstrate and discuss some attemps to solve the
    numerous performance issues that comes with self-hosting a
    language such as Racket on a meta-tracing JIT compiler, such as
    allocation slow-downs caused by deep continuation chains on the
    interpreter, and the problem with big dispatch loops as discused
    above.

    %% Preliminary analyses indicate, however, there might be a
    %% more general problem that subsumes the aformentioned issues, which
    %% is that the amount of information that's been communicated between
    %% the interpreter and the run-time might not be not enough, for
    %% example for the JIT to maneuver around such a dispatch loop
    %% exercising a large number of branches.

    I will vet the runtime efficiency by running Pycket on various
    different benchmark suites against both Racket's C runtime (which
    actually is a generic JIT compiler) and RacketCS to demonstrate
    that the Pycket, while fully hosting Racket, has a competitive
    performance [DOES IT, THOUGH?].
\end{itemize}

I will augment this thesis statement with one of the following research directions:

\begin{enumerate}
  \item this can be done
  \item that can be done
\end{enumerate}
