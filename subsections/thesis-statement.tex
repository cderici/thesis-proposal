\section{Thesis Statement}

My thesis statement is:

\begin{quote}
  A full and efficient implementation of a self-hosting functional
  programming language is possible with a meta-tracing JIT.
\end{quote}

I will support this thesis statement with the following:

\begin{itemize}
  \item \textit{A full implementation of Racket on RPython.} Racket
    recently decided to adopt Chez Scheme as its runtime. To to that
    end, Racket developers made two important implementation decisions
    that (among other things) made Racket a more flexible language to
    re-target: \textit{\textbf{i)}} separated the expansion and the
    compilation that were fused together in the original C
    implementation \textit{\textbf{ii)}} export some functionality
    that's essential for self-hosting (e.g. reader, expander etc.) for
    the runtime to utilize (e.g. to read and expand Racket
    modules). This relieves any targeted runtime from implementing
    those quite large functionalities in terms of both the code size
    and semantics, thereby allowing an easier hosting of the
    language. This approach is used to adopt Chez Scheme and it also
    works for Pycket to fully host the Racket as well. [MOVE THIS TO INTRO]

    I will demonstrate that Pycket has all the necessary runtime
    support (e.g. primitives, data structures, error handling etc.)
    that Racket code assumes and is able to load and evaluate any
    Racket code, thereby being able to also run and pass Racket's own
    test suite, consisting of more than 800.000 test cases (modulo the
    ones that are acceptable corner cases such as error-message
    differences etc.).
  \item \textit{An efficient implementation of Racket on RPython.} I
    will demonstrate the solutions for the numerous performance issues
    that need to be solved to fully host Racket, such as allocation
    slow-downs caused by deep continuation chains on the CEK that
    non-tail calls build up, and the big dispatch loop problem that we
    discused above. Preliminary analyses indicate, however, there
    might be a more general problem that subsumes the aformentioned
    issues, which is that the amount of information that's been
    communicated between the interpreter and the runtime seems to be
    not enough, for example for the JIT to maneuver around such a
    dispatch loop exercising a large number of branches. [MOVE THIS TO INTRO]

    I will vet the runtime efficiency by running Pycket on various
    different benchmark suites against both Racket's C runtime (which
    actually is a generic JIT compiler) and RacketCS to demonstrate
    that the Pycket, while fully hosting Racket, has a competitive
    performance [WILL IT, THOUGH?].
\end{itemize}

I will augment this thesis statement with one of the following research directions:

\begin{enumerate}
  \item this can be done
  \item that can be done
\end{enumerate}
