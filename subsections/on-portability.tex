\vspace{-0.25cm}

\subsubsection{On Racket's Portability}
\label{subsec:portability}

The ability to provide high-level functionalities as callable
functions to the hosting virual machine is one of the key ideas in
improving Racket's portability. A VM that wants to host Racket not
only gets the high-level implementations such as the module system,
macro system for free, but also gets highly re-usable functionalities
that it can integrate the systems it implements, such as the top-level
repl example in \figref{fig:repl-rpython}.

Moreover, Racket implements and exports abstract functionalities that
are tightly coupled with the hosting run-time as well. For instance
the expander linklet provides the \racketcode{eval} function, which is
basically an abstract interpreter for Racket. A VM that hosts Racket
can call \racketcode{eval}, which interprets Racket code by using the
run-time primitives and \racketcode{compile-linklet}
and \racketcode{instantiate-linklet} that are implemented and provided
by the VM itself.

Therefore, one of the essential nuances of self-hosting Racket with
the expander linklet, is that the interaction between the run-time and
the expander is a two-way street. The VM implements and uses
\racketcode{compile-linklet} and \racketcode{instantiate-linklet} to
import the functions provided by the expander, and the expander
provides functions that uses the \racketcode{compile-linklet} and
\racketcode{instantiate-linklet} functions, along with the other
run-time primitives as well.

For example, the \racketcode{dynamic-require} used above (provided by
the expander linklet) is a Racket function that dynamically loads a
Racket module (if it's not already loaded) by resolving the module
path, finding the source code in the file system, reading and
expanding the codes and modifying the namespace registry. The Racket
code inside the expander that implements all that, calls run-time
primitives such as for file-system support and also calls the
\racketcode{compile-linklet} and \racketcode{instantiate-linklet} for
expanding and instantiating all the required
modules. \racketcode{eval} works in a similar fashion as well. This
intertwined nature of the high-level language facilities and the
low-level run-time support is central to the Racket's improved
portability.
