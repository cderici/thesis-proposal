\subsubsection{On Portability}
\label{subsec:portability}

The ability to provide high-level functionalities as callable
functions to a virual machine is the key idea in Racket's
portability. A VM that wants to host Racket not only gets the
high-level implementations such as the module system, macro system for
free, but also gets highly re-usable functionalities that it can
integrate the systems it implements (such as the top-level repl
example in the previous section).

Moreover, Racket implements and exports abstract functionalities that
are tightly coupled with the hosting run-time as well. For instance
the expander linklet provides the \verb|eval| function, which is
basically an abstract interpreter for Racket. A VM that hosts Racket
can call \verb|eval|, which interprets Racket code by using the
primitives and \verb|instantiate-linklet| that are implemented and
provided by the VM itself.

We will discuss a bit more on increasing the language portability in
\secref{subsec:pycket} where I will demonstrate this API in action on
Pycket.
