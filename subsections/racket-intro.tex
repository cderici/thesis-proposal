\subsection[Racket on Chez Scheme]{Racket on Chez Scheme (RacketCS)}
\label{subsec:racketcs}

In 2017, Racket decided to adopt the Chez Scheme as its runtime. To to
that end, they made two important implementation decisions (among
others) that allowed Racket to be more flexible in re-targeting
different VMs: \textit{\textbf{i)}} separated the expansion and the
compilation that were fused together in the original C implementation
\textit{\textbf{ii)}} export some functionality that's essential for
self-hosting (e.g. reader, expander etc.) for the run-time to utilize
(e.g. to read and expand Racket modules). This relieved any targeted
run-time from implementing those functionalities that are quite large
in terms of both code size and semantics, thereby allowing an easier
hosting of the language. This approach is used to adopt the Chez
Scheme and it also made it easier for Racket to be hosted in other
run-times as well \cite{racket-on-chez-19}.
