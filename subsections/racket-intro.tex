\subsection[More Portability for Racket]{More Portability for Racket}
\label{subsec:racketcs}

Racket's original interpreter, as well as the compiler and run-time
(including the built-in primitives and data structures) has been in C
since its first design in 1995. Porting of the GUI layer in 2010 and
the expander in 2016 from C to Racket has surprisingly shown that
Racket is easier to maintain and modify than C. This essentially
started the effort of porting more of Racket from C to Racket, which
in turn, created the need of a more maintainable host run-time
system. In 2017, Racket decided to adopt the Chez Scheme as its
run-time. To to that end, two important implementation decisions
(among others) allowed Racket to be more flexible in re-targeting
different VMs: \textit{\textbf{i)}} separated the expansion and the
compilation that were fused together in the original C implementation
\textit{\textbf{ii)}} export some functionality that's essential for
self-hosting (e.g. reader, expander etc.) for the run-time to utilize
(e.g. to read and expand Racket modules). This relieved any targeted
run-time from implementing those functionalities that are quite large
in terms of both code size and semantics, thereby allowing an easier
hosting of the language. This approach is used to adopt the Chez
Scheme and it also made it easier for Racket to be hosted in other
run-times as well \cite{racket-on-chez-19}.
