\subsection{Racket on Chez}

Racket recently decided to adopt Chez Scheme as its runtime
\cite{racket-on-chez-19}. To to that end, Racket developers made two
important implementation decisions that (among other things) made
Racket a more flexible language to re-target: \textit{\textbf{i)}}
separated the expansion and the compilation that were fused together
in the original C implementation \textit{\textbf{ii)}} export some
functionality that's essential for self-hosting (e.g. reader, expander
etc.) for the runtime to utilize (e.g. to read and expand Racket
modules). This relieves any targeted runtime from implementing those
quite large functionalities in terms of both the code size and
semantics, thereby allowing an easier hosting of the language. This
approach is used to adopt Chez Scheme and it also made it possible for
Pycket to fully host the Racket as well.
