\documentclass[9pt]{extarticle}

\usepackage[margin=1in]{geometry}
%\usepackage{savesym}
%\savesymbol{r}
%\savesymbol{AA}
%\usepackage{esop-common}
%\usepackage{infer-common}

\title{Thesis Proposal: Towards a Rapid and Efficient Implementation
  of a Self-Hosting Programming Language}
\author{Caner Derici}
\usepackage[backend=bibtex]{biblatex}
\addbibresource{bibliography.bib}

\pdfinfo{%
  /Title    (Derici, Thesis Proposal)
  /Author   (Caner Derici)
  /Creator  (Caner Derici)
  /Producer (Caner Derici)
  /Subject  (Thesis Proposal)
  /Keywords (phd, thesis, proposal)
}

\renewcommand*\contentsname{}

\begin{document}



\newcommand\figref[1]{Figure~\ref{#1}}
\newcommand\secref[1]{Section~\ref{#1}}
\newcommand\inputsec[1]{\input{subsections/#1}}

\def\dash {\text{-}}


\maketitle

\begin{abstract}
  \inputsec{abstract}
\end{abstract}

\tableofcontents

\newpage

\section{Introduction}

To contextualize the thread of work I propose in this thesis, this
section gives a general introduction to the rapid implementation of
dynamic programming languages using meta-tracing, and then provides a
general introduction on the problem of self-hosting on meta-tracing.

\inputsec{meta-trace-intro}

\inputsec{racket-intro}

\inputsec{meta-trace-prob}

\inputsec{thesis-statement}

\section{Technical Overview}

%\input{esop-overview}

%\input{esop-formal-model}

%\input{esop-metatheory}

\subsection{RPython Framework}

\subsection{Language Portability via Linklets}

\subsection{Pycket: A Full Racket Implementation}

\subsection{Performance Flaws of Self-Hosting on JIT Compilers}

\subsection{Approaches on Increasing Performance}

\subsubsection{A Second Interpreter in Play}

\subsubsection{Meta-hints for Interpreter-JIT Communication}

\section{Related Work}

% one sentence

% What's diff about TC from the related work
% small summary for diesel....
% - diesel supports x
%  - calculus supports some subset of x
% we support y, which covers most of x but also foo

\section{Research Plan and Timeline}

I have already made progress towards my thesis:

\begin{itemize}
  \item .....
\end{itemize}

To complete my thesis, I plan to follow this timeline:

\begin{itemize}
  \item \textbf{[June-July 2018 - Completed]} this one
    \begin{itemize}
    \item do this
    \end{itemize}
  \item \textbf{[July-August 2018 - Completed]} that one
    \begin{itemize}
      \item do that
    \end{itemize}
  \item \textbf{[January-May 2019]} Write dissertation
  \item \textbf{[June 2019]} Defend
\end{itemize}

\subsection{Publications}

I plan to publish the following papers:

\begin{itemize}
  \item \textbf{2020} \emph{cool paper}
\end{itemize}

\printbibliography

\end{document}
