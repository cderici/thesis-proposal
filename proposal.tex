\documentclass[9pt]{extarticle}

\usepackage[margin=1in]{geometry}
\usepackage{mathtools}
\usepackage{bm}
\usepackage{stmaryrd}
\usepackage{minted}
\usepackage{amssymb}
\usepackage{mdframed}
\usepackage{wrapfig}
\usepackage{subcaption}
\usepackage{listings}
\usepackage{makecell}

\lstset{
  basicstyle=\ttfamily,
  columns=flexible
}

\setcounter{tocdepth}{2}

%\usepackage{savesym}
%\savesymbol{r}
%\savesymbol{AA}
%\usepackage{esop-common}
%\usepackage{infer-common}

\title{Thesis Proposal: Towards Rapid and Efficient Implementation
  for Self-Hosting Functional Programming Languages}
\author{Caner Derici}
\usepackage[backend=bibtex]{biblatex}
\addbibresource{bibliography.bib}

\pdfinfo{%
  /Title    (Derici, Thesis Proposal)
  /Author   (Caner Derici)
  /Creator  (Caner Derici)
  /Producer (Caner Derici)
  /Subject  (Thesis Proposal)
  /Keywords (phd, thesis, proposal)
}

\renewcommand*\contentsname{}

\begin{document}



\newcommand\figref[1]{Figure~\ref{#1}}
\newcommand\secref[1]{Section~\ref{#1}}
\newcommand\inputsec[1]{\input{subsections/#1}}

\def\dash {\text{-}}


\maketitle

\begin{abstract}
  \inputsec{abstract}
\end{abstract}

\tableofcontents

\newpage

\section{Introduction}
\label{sec:intro}

To contextualize the thread of work I propose in this thesis, this
section gives a general introduction to the rapid implementation of
dynamic programming languages using meta-tracing, and then discusses
the increased portability in Racket and provides a general
introduction on the problem of self-hosting on meta-tracing.

\inputsec{meta-trace-intro}
\inputsec{racket-intro}
\inputsec{meta-trace-prob}

\section{Thesis Statement}
\label{sec:thesis}

\inputsec{thesis-statement}

\section{Technical Overview}
\label{sec:technical}

In this section, I will expand on the introduction, and start with
discussing in \secref{subsec:rpython} the RPython framework and how
meta-tracing works. Then in \secref{subsec:performance}, I will
introduce in detail the problems we will attempt to address in this
thesis with preliminary experiments. Next we move on to building our
tool on which we will discuss and demonstrate our solution
approaches. I will start by detailing how language portability is
increased in Racket via the linklets in \secref{subsec:linklets}. Then
I will continue in \secref{subsec:pycket} with explaining how Pycket
incorporates the linklets to become a full implementation of
self-hosting Racket. Finally in \secref{subsec:solutions}, I will
discuss our solution approaches to the performance issues introduced
earlier with experiments and formalism using Pycket.

\inputsec{rpython}
\inputsec{racket}
\inputsec{pycket}
\inputsec{performance-issues}
\inputsec{solution-approaches}

\section{Related Work}
\label{sec:related}

\inputsec{related}

\section{Research Plan and Timeline}
\label{sec:timeline}

\inputsec{timeline}

\printbibliography

\end{document}
