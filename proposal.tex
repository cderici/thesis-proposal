\documentclass[9pt]{extarticle}

\usepackage[margin=1in]{geometry}
%\usepackage{savesym}
%\savesymbol{r}
%\savesymbol{AA}
%\usepackage{esop-common}
%\usepackage{infer-common}

\title{Thesis Proposal: Towards Rapid and Efficient Implementation
  of Self-Hosting Functional Programming Languages}
\author{Caner Derici}
\usepackage[backend=bibtex]{biblatex}
\addbibresource{bibliography.bib}

\pdfinfo{%
  /Title    (Derici, Thesis Proposal)
  /Author   (Caner Derici)
  /Creator  (Caner Derici)
  /Producer (Caner Derici)
  /Subject  (Thesis Proposal)
  /Keywords (phd, thesis, proposal)
}

\renewcommand*\contentsname{}

\begin{document}



\newcommand\figref[1]{Figure~\ref{#1}}
\newcommand\secref[1]{Section~\ref{#1}}
\newcommand\inputsec[1]{\input{subsections/#1}}

\def\dash {\text{-}}
\def\la {\bm{L_\alpha}}
\def\lb {\bm{L_\beta}}

\newcommand\racketcode[1]{{\mintinline[fontsize=\small]{racket}{#1}}}


\maketitle

\begin{abstract}
  \inputsec{abstract}
\end{abstract}

\tableofcontents

\newpage

\section{Introduction}
\label{sec:intro}

To contextualize the thread of work I propose in this thesis, this
section gives a general introduction to the rapid implementation of
dynamic programming languages using meta-tracing, and then provides a
general introduction on the problem of self-hosting on meta-tracing.

\inputsec{meta-trace-intro}
\inputsec{racket-intro}
\inputsec{meta-trace-prob}

\section{Thesis Statement}
\label{sec:thesis}

\inputsec{thesis-statement}

\section{Technical Overview}
\label{sec:technical}

In this section, I will expand on the introduction, and start in
\secref{subsec:rpython} with discussing the RPython framework and how
meta-tracing works. Then I will detail how the increased language
portability is achieved in Racket via linklets in
\secref{subsec:linklets}. I will continue in \secref{subsec:pycket}
with explaining how Pycket can be a full implementation of
Racket. Then in \secref{subsec:performance} I will discuss in detail
the problem we tackle in this thesis, and finally I will talk about
our solution approaches in \secref{subsec:solutions}.

\inputsec{rpython}
\inputsec{linklets}
\inputsec{pycket}
\inputsec{performance-issues}
\inputsec{solution-approaches}

\section{Related Work}
\label{sec:related}

\inputsec{related}

\section{Research Plan and Timeline}
\label{sec:timeline}

\inputsec{timeline}

\printbibliography

\end{document}
